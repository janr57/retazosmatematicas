% coord_R2.tex
%
% Copyright (C) 2022--2026 José A. Navarro Ramón <janr.devel@gmail.com>
% Licencia del código GPLv2
% Licencia Creative Commons Recognition Non-Commercial Share-alike.
% (CC-BY-NC-SA)

\chapter{Sistemas de coordenadas en
  \mathinhead{\symbb{R}^2}{espr2}}
Este capítulo tiene dos objetivos:
\begin{enumerate}
\item Analizar distintos sistemas de coordenadas en el plano $\symbb{R}^2$ y la expresión
  del resultado de distintas operaciones en estos sistema.
\item Introducir elementos básicos de geometría diferencial necesarios para la
  realización del objetivo anterior.
\end{enumerate}

En el capítulo se analizan brevemente unos pocos sistemas de coordenadas en el plano,
seguido de la acción del operador nabla sobre campos escalares y vectoriales

El capítulo comienza recordando conceptos básicos del sistema de coordenadas cartesianas
rectangulares. Cuando digamos coordenadas \emph{cartesianas} sin más, nos referimos a
estas.

Se sigue con una descripción de sistemas de coordenadas cartesianas oblícuas, en adelante, coordenadas \emph{oblícuas}, en contraposición con las \emph{rectangulares},
cuando convenga compararlas, pues ambas son \emph{cartesianas}.

Después nos adentramos en la descripción de las coordenadas polares planas.

Se introducen determinados conceptos de geometría diferencial, como \emph{base natural},
\emph{base unitaria}, \emph{métrica}, \emph{factores de escala},
\emph{vectores y coordenadas contravariantes y covariantes}, \emph{tensores},
\emph{espacio dual}, etc., conforme se van necesitando, unos con más detalle que otros,
de manera que, al final del capítulo, se termina con una idea básica de esta rama de las
matemáticas. Lo nos interesa ahora es que, con estos conocimientos, se podrán seguir los
siguientes capítulos que versen sobre otros sistemas de coordenadas. Por descontado, la
geometría diferencial es mucho más que lo que aquí se presenta, pues solo se ha arañado
parte de la superficie.

\section{Coordenadas cartesianas rectangulares}
Vector de posición de cualquier punto del espacio $\symbb{R}^2$ en coordenadas cartesianas
\begin{equation}\label{eq:R2-r_rect}
  \vvv{r}
  = x\,\uvec{\i} + y\,\uvec{\j}
\end{equation}
\begin{figure}[ht]
  % Escala
  \def\scl{1}
  %
  \pgfmathsetmacro{\EJESEXTRA}{-0.2}
  \pgfmathsetmacro{\EJESLONG}{2.3}
  % Eje x
  \pgfmathsetmacro{\XMLONG}{\EJESEXTRA}
  \pgfmathsetmacro{\XPLONG}{\EJESLONG}
  % Eje y
  \pgfmathsetmacro{\YMLONG}{\EJESEXTRA}
  \pgfmathsetmacro{\YPLONG}{\EJESLONG}
  % Versores
  \pgfmathsetmacro{\VMOD}{0.6}
  % Vector P
  \pgfmathsetmacro{\PMOD}{2.5}
  \pgfmathsetmacro{\PANG}{40}
  % Fondo
  \pgfmathsetmacro{\HORZ}{0.25}
  \pgfmathsetmacro{\VERT}{0.25}
  % 
  \centering
  \begin{tikzpicture}[%
    scale=\scl,
    every node/.style={black,font=\small},
    eje/.style={->},
    proyeccion/.style={black!20, line width=0.4},
    vector/.style={-{Latex[round]}, shorten >=1.2pt, line width=.8pt,\colorV},
    versor/.style={-{Latex[round, width=4pt, length=6pt]}, line width=1pt, \colorVred},
    textooriginal/.style={\colorTorig},
    pcirculo/.style={fill=\colorPorig, draw=black},
    angulogirado/.style={fill=\colorAngGreen, draw=\colorGirodos},
    background/.style={%
      line width=\bgborderwidth,
      draw=\bgbordercolor,
      fill=\bgcolor,
    },
    ]
    % COORDENADAS
    % Origen
    \coordinate (O) at (0,0);
    % Extremo izdo del eje x e inferior del eje y
    \coordinate (xini) at (\XMLONG cm,0);
    \coordinate (yini) at (0,\YMLONG cm);
    \coordinate (ifin) at (\VMOD cm, 0);
    \coordinate (jfin) at (0, \VMOD cm);
    % Eje x
    \path[save path=\ejex] (xini) -- coordinate[pos=0.55] (xtexto) (\XPLONG cm, 0)
    coordinate (xfin);
    % Eje y
    \path[save path=\ejey] (yini) -- coordinate[pos=0.55] (ytexto) (0, \YPLONG cm)
    coordinate (yfin);
    % Versores i y j
    \path[save path=\i] (O) -- coordinate[pos=0.5] (itexto) (ifin);
    \path[save path=\j] (O) -- coordinate[pos=0.45] (jtexto) (jfin);    
    % Vector
    \path[save path=\vector] (O) -- coordinate[pos=0.5] (pmid) (\PANG:\PMOD cm)
    coordinate (P);
    % Proyección de P en los ejes
    \path[save path=\proyx] (P) |- (xfin);
    \path[save path=\proyy] (P) -| (yfin);

    % DIBUJOS
    % Proyección de P en los ejes
    \draw[proyeccion, use path=\proyx];
    \draw[proyeccion, use path=\proyy];
    % Ejes
    % x
    \draw[eje, use path=\ejex];
    \node[right] at (xfin) {$x$};
    \node[below] at (xtexto) {$x$};
    % y
    \draw[eje, use path=\ejey];
    \node[above] at (yfin) {$y$};
    \node[left] at (ytexto) {$y$};
    % Versores
    \draw[versor, use path=\i];
    \node[below, \colorTred] at (itexto) {\scriptsize $\uvec{\i}$};
    \draw[versor, use path=\j];
    \node[left, \colorTred] at (jtexto) {\scriptsize $\uvec{\j}$};
    % Vector y punto  P
    \draw[vector, use path=\vector];
    \node[above left=-3pt and -1pt] at (pmid) {$\vvv{r}$};
    \filldraw[fill=green, draw=black] (P) circle[radius=1.2pt];
    \node[above right=0pt and -5pt] at (P) {$P(x,y)$};
    
    % Origen
    \filldraw (O) circle [radius=.2pt];
    % Fondo amarillo
    \coordinate (SW) at ($(current bounding box.south west) + (-\HORZ cm,-\VERT cm)$);
    \coordinate (NE) at ($(current bounding box.north east) + (\HORZ cm,\VERT cm)$);    
    \begin{scope}[on background layer]
      \draw[background] (SW) rectangle (NE);
    \end{scope}
  \end{tikzpicture}
  \caption{Coordenadas cartesianas de un punto $P$ del plano (en verde) y vectores
    unitarios (en rojo). Se puede comprobar que el vector de posición del punto es
    $\vvv{r} = x\uvec{i} + y\uvec{j}$.}
  \label{fig:R2-coord_rect-ij}
\end{figure}

\subsection{Base natural, vectores unitarios y su producto escalar}
\subsubsection{Base natural y vectores unitarios}
Si derivamos el vector de posición anterior \eqref{eq:R2-r_rect}, con respecto a
cada una de las coordenadas, obtenemos los vectores de una  posible base en estas
coordenadas. Estos vectores llevan información concreta acerca de cómo varia la posición
en el espacio cuando variamos cada una de las coordenadas ---tal y como sugieren las
derivadas---, lo que será importante para obtener la métrica en estas coordenadas.

Como
\begin{equation}
  \vvv{r} = x\uvec{\i} + y\uvec{\j}
\end{equation}

Entonces, derivando se obtiene
\begin{subequations}
  \begin{align}\label{eq:R2-cart-ex}
    \xhat{e}_x
    &=
      \dfrac{\partial\vvv{r}}{\partial x}
      = \uvec{\i}\\
    \label{eq:R2-cart-ey}
    \xhat{e}_y
    &= \dfrac{\partial\vvv{r}}{\partial y}
      = \uvec{\j}
  \end{align}
\end{subequations}
Estos vectores forman la denominada \emph{base natural} en cartesianas que, en este
sistema coincide con los vectores unitarios tradicionales $\uvec{\i}$ y $\uvec{\j}$,
aunque en otros sistemas de coordenadas no tienen por qué ser unitarios
\begin{equation}\label{eq:R2-cart-base}
  B=\set{\xhat{e}_x, \xhat{e}_y} = \set{\uvec{\i},\, \uvec{\j}}
\end{equation}

La métrica en cada sistema de coordenadas define la estructura del espacio, en
particular, sin ella no se pueden medir ni distancias ni ángulos. Es como si
proporcionara un sistema de reglas y goniómetros con los cuales medir.

De la equivalencia de la base natural con la unitaria, se puede inferir que la métrica en
coordenadas cartesianas rectangulares en $\symbb{R}^2$ se representará como una matriz
unidad (en el apartado \ref{ssect:R2-cart-metrica-factores_escala} se calculará con detalle).

\subsection{Métrica y factores de escala}
\label{ssect:R2-cart-metrica-factores_escala}

\subsubsection{Producto escalar de los vectores de la base natural}
Los vectores de la base fundamental son vectores unitarios.
La base normalizada en coordenadas cartesianas en este espacio está formada por los vectores $\uvec{\i}$ y $\uvec{\j}$, en los ejes $x$ e $y$, respectivamente.
A continuación resumimos el valor de los productos escalares entre ellos, tal y como
aprendimos en cálculo vectorial fundamental en coordenadas cartesianas o en física
elemental
\begin{subequations}
  \begin{align}
    \label{eq:R2-cart-ii_prod_escalar}
    \xhat{e}_x\cdot\xhat{e}_x = \uvec{\i}\cdot\uvec{\i}
    &= |\uvec{\i}|\,|\uvec{\i}|\cos 0 = 1\\
    \label{eq:R2-cart-ij_prod_escalar}    
    \xhat{e}_x\cdot\xhat{e}_y = \uvec{\i}\cdot\uvec{\j}
    &= |\uvec{\i}|\,|\uvec{\j}|\cos\pi/2 = 0\\
    \label{eq:R2-cart-ji_prod_escalar} 
    \xhat{e}_y\cdot\xhat{e}_x = \uvec{\j}\cdot\uvec{\i}
    &= |\uvec{\j}|\,|\uvec{\i}|\cos\pi/2 = 0\\
    \label{eq:R2-cart-jj_prod_escalar}
    \xhat{e}_y\cdot\xhat{e}_y = \uvec{\j}\cdot\uvec{\j}
    &= |\uvec{\j}|\,|\uvec{\j}|\cos 0 = 1
  \end{align}
\end{subequations}

\subsubsection{Métrica y factores de escala}
La métrica en cartesianas se calcula utilizando los vectores de la base natural,
calculada en \eqref{eq:R2-cart-ex} y \eqref{eq:R2-cart-ey}.
Los productos escalares se calcularon en \eqref{eq:R2-cart-ii_prod_escalar},
\eqref{eq:R2-cart-ij_prod_escalar}, \eqref{eq:R2-cart-ji_prod_escalar} y
\eqref{eq:R2-cart-jj_prod_escalar}
\begin{equation}\label{eq:R2-cart-metrica}
  \mmm{g}
  = g_{ij}
  = \begin{pmatrix}
    g_{11} & g_{12}\\
    g_{21} & g_{22}
    \end{pmatrix}
  = \begin{pmatrix}
      \xhat{e}_x\cdot\xhat{e}_x & \xhat{e}_x\cdot\xhat{e}_y\\
      \xhat{e}_y\cdot\xhat{e}_x & \xhat{e}_y\cdot\xhat{e}_y
  \end{pmatrix}
  = \begin{pmatrix}
    1 & 0\\
    0 & 1
  \end{pmatrix}
\end{equation}

La raíz cuadrada de los elementos diagonales de la métrica en un sistema de coordenadas ortogonales, como el que nos ocupa, se llaman \emph{factores de escala} o \emph{coeficientes métricos}. En coordenadas cartesianas rectangulares serían
\begin{align*}
  h_x &= \sqrt{g_{11}} = \sqrt{1} = 1\\
  h_y &= \sqrt{g_{22}} = \sqrt{1} = 1  
\end{align*}

Fijémonos en la métrica \eqref{eq:R2-cart-metrica}:
\begin{itemize}
\item Los elementos no diagonales son nulos. Esto se interpreta como que los vectores
  $\xhat{e}_x$ y $\xhat{e}_y$ son independientes, esto es, son ortogonales.
  Esto implica que un cambio en una coordenada no afecta a la otra. Así, si nos
  desplazamos una distancia cualquiera en el eje $x$, no habremos recorrido ninguna
  distancia en el eje $y$, y viceversa, por lo que $h_{xy}=h_{yx}=0$.
\item Los elementos diagonales valen la unidad. Cada elemento diagonal de la métrica que
  vale la unidad implica que un cambio en la coordenada respectiva implica una distancia
  recorrida de una unidad en el espacio. En realidad, el espacio recorrido es el factor
  de escala correspondiente (la raíz cuadrada del elemento diagonal correspondiente),
  pero en este caso da lo mismo
  \begin{subequations}
    \begin{align}
      \Delta s &= h_x \Delta x = \sqrt{1}\, \Delta x = \Delta x\\
      \Delta s &= h_y \Delta y = \sqrt{1}\, \Delta y = \Delta y
    \end{align}
  \end{subequations}
  Para adquirir algo de intuición, imaginemos que nos desplazamos cualquier distancia en
  el eje $x$, digamos un metro; es fácil entender que en el espacio nos habremos
  desplazado esa misma distancia. Por tanto es lógico razonar que el factor de escala
  $h_x$ vale 1. Lo mismo podemos decir para el factor de escala en el eje $y$, $h_y=1$.
\end{itemize}

\subsubsection{Revisión del producto escalar}
Dijimos que la métrica nos permite calcular distancias. En concreto, el cuadrado del
módulo de un vector $\vvv{A}$ se mide realizando el producto escalar del vector por sí
mismo.
El producto escalar se encarga de esto, y se puede ampliar mediante el cálculo
matricial, utilizando la métrica (en otros sistemas de coordenadas veremos que es más
importante, mientras que en cartesianas se puede obviar)
\[
  \vvv{v}\cdot\vvv{w}
  =
  \vvv{v}^\transp \mmm{g} \vvv{w}
  =
  \begin{pmatrix}
    v_x &  v_y
  \end{pmatrix}
  \begin{pmatrix}
    1 & 0\\
    0 & 1
  \end{pmatrix}
  \begin{pmatrix}
    w_x\\
    w_y
  \end{pmatrix}
  =
  v_xw_x + v_y w_y
\]
  
Así, el cuadrado del módulo de un vector en cartesianas es
\[
  v^2
  =
  \|\vvv{v}\|^2
  =
  \vvv{v}\cdot\vvv{v}
  =
  \vvv{v}^\transp \mmm{g} \vvv{v}
  =
  \begin{pmatrix}
  v_x &  v_y
\end{pmatrix}
\begin{pmatrix}
  1 & 0\\
  0 & 1
\end{pmatrix}
\begin{pmatrix}
  v_x\\
  v_y
\end{pmatrix}
=
v_x^2 + v_y^2
\]
que coincide con el valor que obteníamos en el cálculo vectorial básico.

\subsection{Gradiente, divergencia y laplaciana}
Presentamos un resumen de estas operaciones, tal y como se presentan en un curso básico
de cálculo
\begin{itemize}
\item El gradiente de un campo escalar $\phi(x,y)$ es un vector
\begin{equation}
  \vvv{\nabla}\phi
  = \dfrac{\partial \phi}{\partial x}\,\uvec{\i}
  + \dfrac{\partial \phi}{\partial y}\,\uvec{\j}
  = \begin{pmatrix}
    \partial \phi/\partial x\\
    \partial \phi/\partial y
    \end{pmatrix}
\end{equation}
El vector gradiente de $\phi$ evaluado en un punto genérico $(x,y)$ del dominio de $\phi$
indica la dirección en la cual el campo $\phi$ varía más rápidamente y su módulo
representa el ritmo de variación de $\phi$ en la dirección de dicho vector gradiente.

El operador gradiente se comporta como un vector bajo rotaciones, decimos esto porque
cuando opera frente a un campo escalar, se obtiene otro formado por vectores
\begin{equation}
  \vvv{\nabla}
  = \dfrac{\partial}{\partial x}\,\uvec{\i}
  + \dfrac{\partial}{\partial y}\,\uvec{\j}
  = \begin{pmatrix}
    \partial/\partial x\\
    \partial/\partial y
    \end{pmatrix}
\end{equation}

\item La divergencia de un campo vectorial $\vvv{A}=(A_x, A_y)$ es un escalar
  \begin{equation}\label{eq:R2-divergencia-cartesianas}
    \vvv{\nabla}\cdot\vvv{A}
    = \begin{pmatrix}
      \partial/\partial x & \partial/\partial y      
    \end{pmatrix}
    \begin{pmatrix}
      1 & 0\\
      0 & 1
    \end{pmatrix}
    \begin{pmatrix}
      A_x\\
      A_y
    \end{pmatrix}
    = \dfrac{\partial A_x}{\partial x}
    + \dfrac{\partial A_y}{\partial y}
  \end{equation}
  La divergencia mide la diferencia entre el flujo saliente y el flujo entrante de un
  campo vectorial sobre la superficie que rodea a un volumen de control, por tanto, si el
  campo tiene ''fuentes'' la divergencia será positiva, y si tiene "sumideros", la
  divergencia será negativa. La divergencia mide la rapidez neta con la que se conduce la
  materia al exterior de cada punto, y en el caso de ser la divergencia idénticamente
  igual a cero, describe al flujo incompresible del fluido, llamado también campo
  solenoidal.  
\item La laplaciana de un campo escalar $f(x,y)$ es la divergencia del gradiente y es un
  escalar
  \begin{equation}\label{eq:R2-gradiente-cartesianas}
    \Delta \phi
    =
    \nabla^2 \phi
    = \vvv{\nabla}\cdot\vvv{\nabla}\phi
    = \dfrac{\partial^2 \phi}{\partial x^2}
    + \dfrac{\partial^2 \phi}{\partial y^2}
  \end{equation}
Mide la concentración de un campo en un punto. Si es negativo, indica una tendencia a la
concentración (fuente); si es positivo, indica dispersión.

El operador laplaciano es
\begin{equation}
  \nabla^2
  = \dfrac{\partial^2}{\partial x^2}
  + \dfrac{\partial^2}{\partial y^2}
\end{equation}
\end{itemize}

\subsection{Elemento de área}
El elemento diferencial de área en cartesianas es
\begin{equation}
  d^2a = dx\,dy
\end{equation}
\begin{figure}[ht]
  % Escala
  \def\scl{1}
  %
  \pgfmathsetmacro{\EJESEXTRA}{-0.4}
  \pgfmathsetmacro{\EJESLONG}{2.3}
  % Eje x
  \pgfmathsetmacro{\XMLONG}{\EJESEXTRA}
  \pgfmathsetmacro{\XPLONG}{\EJESLONG}
  % Eje y
  \pgfmathsetmacro{\YMLONG}{\EJESEXTRA}
  \pgfmathsetmacro{\YPLONG}{\EJESLONG}
  % dx, dy
  \pgfmathsetmacro{\LONG}{0.9}
  \pgfmathsetmacro{\DX}{\LONG}
  \pgfmathsetmacro{\DY}{\LONG}
  % Posición del elemento de área
  \pgfmathsetmacro{\PX}{0.5}
  \pgfmathsetmacro{\PY}{1.0}
  % Fondo
  \pgfmathsetmacro{\HORZ}{0.25}
  \pgfmathsetmacro{\VERT}{0.25}
  % 
  \centering
  \begin{tikzpicture}[%
    scale=\scl,
    every node/.style={black,font=\small},
    eje/.style={->},
    proyeccion/.style={black!20, line width=0.4},
    area/.style={fill=green!40, draw=black!20},
    %vector/.style={-{Latex[round]}, shorten >=1.2pt, line width=.8pt,\colorV},
    %versor/.style={-{Latex[round, width=4pt, length=8pt]}, line width=1pt, \colorVred},
    textooriginal/.style={\colorTorig},
    %pcirculo/.style={fill=\colorPorig, draw=black},
    %angulogirado/.style={fill=\colorAngGreen, draw=\colorGirodos},
    background/.style={%
      line width=\bgborderwidth,
      draw=\bgbordercolor,
      fill=\bgcolor,
    },
    ]
    % COORDENADAS
    % Origen
    \coordinate (O) at (0,0);
    % Extremo izdo del eje x e inferior del eje y
    \coordinate (xini) at (\XMLONG cm,0);
    \coordinate (yini) at (0,\YMLONG cm);
    % Eje x
    \path[save path=\ejex] (xini) -- (\XPLONG cm, 0) coordinate (xfin);
    % Eje y
    \path[save path=\ejey] (yini) -- (0, \YPLONG cm) coordinate (yfin);
    % Punto P
    \coordinate (P) at (\PX, \PY);
    %
    \path (P) -- coordinate[pos=0.5] (dxtext) ++(\DX, 0) coordinate (PDX);
    \path (P) -- coordinate[pos=0.5] (dytext) ++(0, \DY) coordinate (PDY);
%    % Vector
%    \path[save path=\vector] (O) -- coordinate[pos=0.5] (pmid) (\PANG:\PMOD cm)
%    coordinate (P);
%    % Proyección de elemento de área en los ejes
    \path[save path=\proyPx] (P) |- coordinate (Px) (xfin);
    \path[save path=\proyPy] (P) -| coordinate (Py) (yfin);
    \path[save path=\proyPhorx] (PDX) |- coordinate (Phorx) (xfin);
    \path[save path=\proyPverty] (PDY) -| coordinate (Pverty) (yfin);
    % Texto de las proyecciones en los ejes
    \path (Px) -- coordinate[pos=0.5] (dxtexto) (Phorx);
    \path (Py) -- coordinate[pos=0.5] (dytexto) (Pverty);
    
    % DIBUJOS
    % Proyección de elemento de área en los ejes
    \draw[proyeccion, use path=\proyPx];
    \draw[proyeccion, use path=\proyPy];
    \draw[proyeccion, use path=\proyPhorx];
    \draw[proyeccion, use path=\proyPverty];
    % Texto dx y dy en los ejes
    \node[below] at (dxtexto) {$dx$};
    \node[left] at (dytexto) {$dy$};
    
    % Ejes
    % x
    \draw[eje, use path=\ejex];
    \node[right] at (xfin) {$x$};
    %\node[below] at (xtexto) {$x$};
    % y
    \draw[eje, use path=\ejey];
    \node[above] at (yfin) {$y$};
    % \node[left] at (ytexto) {$y$};
    % Punto P
    % \fill (P) circle[radius=1.2pt];
    % Elemento de área
    \filldraw[area] (P) rectangle coordinate (centro) ++(\DX, \DY);
    \node at (centro) {\footnotesize $d^2a$};

    % Origen
    \filldraw (O) circle [radius=.2pt];

    % Fondo amarillo
    \coordinate (SW) at ($(current bounding box.south west) + (-\HORZ cm,-\VERT cm)$);
    \coordinate (NE) at ($(current bounding box.north east) + (\HORZ cm,\VERT cm)$);    
    \begin{scope}[on background layer]
      \draw[background] (SW) rectangle (NE);
    \end{scope}
  \end{tikzpicture}
  \caption{Elemento de área en coordenadas cartesianas, $d^2a = dx dy$.}
  \label{fig:R2-rect-elemento_area}
\end{figure}

Este elemento se relaciona trivialmente con el jacobiano de la
\emph{transformación identidad} $f(x,y) \longrightarrow f(x,y)$
\[
  d^2a = \dfrac{\partial(x,y)}{\partial(x,y)} dx dy
  = \begin{vmatrix}
    \partial x/\partial x & \partial x/\partial y\\
    \partial y/\partial x & \partial y/\partial y
  \end{vmatrix}
  dx dy
  = \begin{vmatrix}
    1 & 0\\
    0 & 1
  \end{vmatrix}
  dx dy
  = 1\,dx dy
  = dxdy
\]

\section{Coordenadas cartesianas oblícuas}
En esta sección encontraremos ...

\subsection{Base natural}
En coordenadas oblícuas es importante el ángulo que forman los dos ejes. Llamaremos
$\alpha$ a este ángulo. Los ejes oblícuos los representamos por $\tilde{x}^1$ y
$\tilde{x}^2$. La base natural está formada por los vectores unitarios $\xhat{e}_1$ y
$\xhat{e}_2$
\begin{equation}\label{eq:R2-oblic_base}
  B' = \set{\xhat{e}_1, \xhat{e}_2}
\end{equation}

\subsection{Cambio de base}
Para relacionar esta base $B'$ con la estándar de las rectilíneas
$B=\set{\uvec{\i},\uvec{\j}}$, haremos coincidir el versor $\xhat{e}_1$ con $\uvec{\i}$,
esto es, el eje $\tilde{x}^1$ coincide con el eje $x$ de las cartesianas. Véase la figura
\ref{fig:R2-coord_oblicuas_e1e2}.
\begin{figure}[ht]
  \centering
  % Escala
  \def\scl{1.00}
  % 
  % \pgfmathsetmacro{\EJESCARTLONG}{3.0}
  % \pgfmathsetmacro{\EJEOBLICLONG}{2.7}
  \pgfmathsetmacro{\EJESCARTLONG}{2.5}
  \pgfmathsetmacro{\EJEOBLICLONG}{2.1}
  % Eje x
  \pgfmathsetmacro{\XLONG}{\EJESCARTLONG}
  % Eje y
  \pgfmathsetmacro{\YLONG}{\EJESCARTLONG}
  % Eje xtilde
  \pgfmathsetmacro{\XTILDEANG}{0}
  \pgfmathsetmacro{\XTILDELONG}{\EJEOBLICLONG}    
  % Eje ytilde
  \pgfmathsetmacro{\YTILDEANG}{70}
  \pgfmathsetmacro{\YTILDELONG}{\EJESCARTLONG}
  % Versores
  \pgfmathsetmacro{\VTILDEMOD}{1.4}
  \pgfmathsetmacro{\VXTILDEANG}{0}
  \pgfmathsetmacro{\VYTILDEANG}{\YTILDEANG}
  % Vector P
  \pgfmathsetmacro{\PMOD}{2.3}
  \pgfmathsetmacro{\PANG}{40}
  % Fondo
  \pgfmathsetmacro{\HORZ}{0.25}
  \pgfmathsetmacro{\VERT}{0.25}
  % 
  \centering
  \begin{tikzpicture}[%
    scale=\scl,
    every node/.style={black,font=\small},
    ejecart/.style={->, black!30},
    ejeoblic/.style={->, black},
    proyeccion/.style={black!20, line width=0.4},
    % longitud/.style={line width=.5pt,\colorV},
    texto/.style={black,font=\footnotesize},
    textoapagado/.style={\colorTorig},
    vtilde/.style={%
      -{Latex[round,width=4pt,length=6pt]}, line width=1pt, \colorVred},
    pcirculo/.style={fill=\colorPorig, draw=black},
    angulo/.style={fill=green!15, draw=\colorGirodos},
    background/.style={%
      line width=\bgborderwidth,
      draw=\bgbordercolor,
      fill=\bgcolor,
    },
    ]
    % COORDENADAS
    % Origen
    \coordinate (O) at (0,0);
    % Eje x
    \path[save path=\ejex] (O) -- (\XLONG cm, 0) coordinate (xfin);
    % Eje y
    \path[save path=\ejey] (O) -- (0, \YLONG cm) coordinate (yfin);
    % Eje xtilde
    \path[save path=\ejextilde] (O) -- ++(\XTILDEANG:\XTILDELONG cm)
    coordinate (xtildefin);
    % Eje ytilde
    \path[save path=\ejeytilde] (O) -- ++(\YTILDEANG:\YTILDELONG cm)
    coordinate (ytildefin);
    % Versores tilde
    \path[save path=\vxtilde] (O) -- coordinate[pos=1.0] (vxtildetxt)
    ++(\VXTILDEANG:\VTILDEMOD cm);
    \path[save path=\vytilde] (O) -- coordinate[pos=1.1] (vytildetxt)
    ++(\VYTILDEANG:\VTILDEMOD cm) coordinate (e2fin);
    % Proyección de P en los ejes
    % Con eje x
    \path[save path=\proyx] (e2fin) |- (xfin) coordinate[pos=0.5] (e2x) (xfin);
    % Posición de texto en eje x
    \path (O) -- coordinate[pos=0.5] (extxt) (e2x);
    % Con eje y
    \path[save path=\proyy] (e2fin) -| coordinate[pos=0.5] (e2y) (yfin);
    % Posición de texto en eje y
    \path (O) -- coordinate[pos=0.5] (eytxt) (e2y);
    
    % DIBUJOS
    % Ángulo theta
    \path (xfin) -- (O) -- (ytildefin) pic
    [angulo, "\footnotesize $\alpha$",angle radius=12pt, angle eccentricity=0.68]
    {angle = xfin--O--ytildefin};
    
    % Proyección de etilde2 en los ejes y texto correspondiente en ejes x e y
    \draw[proyeccion, use path=\proyx];
    \node[below] at (extxt) {\footnotesize $\cos\alpha$};
    \draw[proyeccion, use path=\proyy];
    \node[rotate=90, below left=4pt and 11pt] at (e2y) {\footnotesize $\sin\alpha$};
    
    % Ejes
    % x
    \draw[ejecart, use path=\ejex];
    \node[textoapagado, below=3pt] at (xfin) {\small $x$};
    % \node[below] at (xtexto) {$r\cos\theta$};
    % y
    \draw[ejecart, use path=\ejey];
    \node[above, textoapagado] at (yfin) {$y$};
    % xtilde
    \draw[ejeoblic, use path=\ejextilde];
    \node[texto, below] at (xtildefin) {\footnotesize $\tilde{x}^1$};
    % ytilde
    \draw[ejeoblic, use path=\ejeytilde];
    \node[above right=0pt and -4pt, texto] at (ytildefin) {$\tilde{x}^2$};
    % \node[rotate=90, left=8pt] at (ytexto) {$r\sin\theta$};
    % Versores
    \draw[vtilde, use path=\vxtilde];
    \node[\colorTred, below] at (vxtildetxt) {\footnotesize $\xhat{e}_1$};
    \draw[vtilde, use path=\vytilde];
    \node[\colorTred,above left=-4pt and -1.5pt] at (vytildetxt)
    {\footnotesize $\xhat{e}_2$};
    % Origen
    \filldraw (O) circle [radius=.2pt];
    
    % Fondo amarillo
    \coordinate (SW) at ($(current bounding box.south west) + (-\HORZ cm,-\VERT cm)$);
    \coordinate (NE) at ($(current bounding box.north east) + (\HORZ cm,\VERT cm)$);
    \begin{scope}[on background layer]
      \draw[background] (SW) rectangle (NE);
    \end{scope}
  \end{tikzpicture}
  \caption{Sistema de coordenadas cartesianas oblícuas que forman un ángulo $\alpha$
    entre sí. También se representan los ejes cartesianos $x$ e $y$ tradicionales, como
    referencia en tono algo apagado. Por último se aprecian en rojo los vectores
    unitarios $\xhat{e}_1$ y $\xhat{e}_2$ del sistema oblícuo y sus proyecciones
    en los ejex $x$ e $y$.}
  \label{fig:R2-coord_oblicuas_e1e2}
\end{figure}

Vamos a expresar estos vectores en función de los $\uvec{\i}$ y $\uvec{\j}$ de las
cartesianas rectangulares
\begin{align}
  \label{eq:R2-e1-ij}
  \xhat{e}_1 &= 1\uvec{\i} + 0\uvec{\j} = \uvec{\i}\\
  \label{eq:R2-e2-ij}
  \xhat{e}_2 &= \cos{\alpha}\, \uvec{\i} + \sin{\alpha}\, \uvec{\j}
\end{align}
Nótese que los vectores de la base son unitarios, por esta razón los habíamos
representado, ya desde el principio, con el acento circunflejo
\begin{align}
  \|\xhat{e}_1\|
  &=
    \|\uvec{\i}\| = 1\\
  \|\xhat{e}_2\|
  &=
    (\cos\theta\,\uvec{\i}+\sin\theta\,\uvec{\j})^2
    = \cos^2\theta + \sin^2\theta = 1
\end{align}

La matriz que permite obtener la nueva base $B'$ a partir de la antigua $B$, está formada
por las componentes de los vectores $(1,0)$ y $(\cos\alpha,\sin\alpha)$ obtenidos en las
expresiones \eqref{eq:R2-e1-ij} y \eqref{eq:R2-e2-ij}, dispuestas en columnas
\[
  \mmm{M}
  =
  \begin{pmatrix}
    1 & \cos\alpha\\
    0 & \sin\alpha
  \end{pmatrix}
\]

Utilizamos esta matriz para obtener los vectores de la base oblícua a partir de la
rectangular. Es importante darse cuenta que los vectores se representan en filas
\[
  \begin{pmatrix}
    \xhat{e}_1 &  \xhat{e}_2
  \end{pmatrix}
  =
  \begin{pmatrix}
    \uvec{\i} & \uvec{\j}
  \end{pmatrix}
  \,
  \begin{pmatrix}
    1 & \cos\alpha\\
    0 & \sin\alpha
  \end{pmatrix}
\]

\subsubsection{Resumiendo el cambio de base}
Como dijimo, la matriz de cambio de base de $B$ a $B'$ está formada por las componentes
de los vectores de la expresión dispuestos en columnas. A continuación se representan, esta matriz de cambio de base y su inversa
\begin{subequations}
  \begin{align}
    &\mmm{M}
      =
      \begin{pmatrix}
        1 & \cos\alpha\\
        0 & \sin\alpha
      \end{pmatrix}\\
    &\mmm{M}^{-1}
      =
      \dfrac{1}{\sin\alpha}
      \begin{pmatrix}
        \sin\alpha & -\cos\alpha\\
        0 & 1        
      \end{pmatrix}
      =
      \begin{pmatrix}
        1 & -\cot\alpha\\
        0 & \phantom{-}\csc\alpha
      \end{pmatrix}
  \end{align}
\end{subequations}
Utilizaremos la matriz directa $M$ para transformar la base rectangular $B$ en la nueva
base oblícua $B'$. En cambio, para obtener la base rectangular a partir de la oblícua,
utilizaríamos la matriz inversa $\mmm{M}^{-1}$. Nótese que los vectores se representan en
filas
\begin{subequations}
\begin{align}
  &\begin{pmatrix}
    \xhat{e}_1 &  \xhat{e}_2
  \end{pmatrix}
  =
    \begin{pmatrix}
      \uvec{\i} & \uvec{\j}
    \end{pmatrix}
    \,
    \begin{pmatrix}
      1 & \cos\alpha\\
      0 & \sin\alpha
    \end{pmatrix}\\
  &\begin{pmatrix}
    \uvec{\i} & \uvec{\j}
  \end{pmatrix}
  =
    \begin{pmatrix}
      \xhat{e}_1 & \xhat{e}_2
    \end{pmatrix}
    \,
    \begin{pmatrix}
      1 & -\cot\alpha\\
      0 & \phantom{-}\csc\alpha
    \end{pmatrix}
    \,
\end{align}
\end{subequations}


\subsection{Componentes de los vectores}
Cuando un vector se expresa en una determinada base, se obtiene una expresión de ese
vector en función de unas determinadas componentes. Por ejemplo, en dos dimensiones y
tomando una base genérica $\set{\vvv{b}_1,\vvv{b}_2}$, escribiríamos
\[
  \vvv{v} = A_1\vvv{b}_1 + A_2\vvv{b}_2
\]
que, como sabemos significa que el vector $\vvv{v}$ se puede representar sumando
los productos de cada cada componente por el vector correspondiente de la base.

Si la base estuviera formada por muchos vectores, o este número no estuviera definido,
entonces sería conveniente utilizar un sumatorio, como
\[
  \vvv{v} = \sum_{i=1}^n A_i\vvv{b}_i
\]

Cuando se opera con expresiones un tanto largas, se suele utilizar la notación de
Einstein, donde se prescinde del sumatorio y se tiene en cuenta que cuando en un
monomio se encuentra el mismo índice, por ejemplo $i$, se debe sumar para todos los
valores de ese índice
\[
  \vvv{v} = \sum_i A_i\vvv{b}_i = A_i \vvv{b}_i
\]

Otro aspecto importante es notar que las componentes de un vector dependen de la base
elegida. Si se cambia de base, las componentes varían. Por ejemplo, si utilizáramos
la base $\set{\vvv{\tilde{b}}_1, \vvv{\tilde{b}}_2}$, las nuevas componentes serían las $\tilde{A}_i$
\[
  \vvv{v} = \tilde{A}_1\vvv{\tilde{b}}_1 + \tilde{A}_2\vvv{\tilde{b}}_2
\]
Cambian las componentes, pero el vector tiene que ser el mismo, el vector no cambia, lo
hace su representación dependiendo de la base utilizada.

\subsection{Componentes contravariantes}
Para seguir, vamos a poner un ejemplo concreto. Supongamos que medimos un cierto vector
y para ello, por sencillez, utilizamos la base estándar en coordenadas cartesianas
rectangulares en metros $B=\set{\uvec{\i}, \uvec{\j}}$, y obtenemos que mide
\qty{1}{\metre} en el eje $x$ y \qty{2}{\metre} en el eje $y$. El vector se podrá
representar como
\[
  \vvv{v} = 1\,\uvec{\i} + 2\,\uvec{\j}\,\unit{\metre}
\]
esto es, las componentes serían $(1, 2)\,\unit{\metre}$.

Supongamos también que deseamos cambiar la base elegida, a otra
$B' = \set{\uvec{\i}' + \uvec{\j}'}$ que mide en centímetros. La nueva base
\emph{es más pequeña} (``$B' < B$'') que la antigua (centímetros en lugar de metros). Pero, para que el vector $\vvv{v}$ no cambie, los valores medidos en centímetros deben
\emph{aumentar} con respecto a las componentes en metros ($A'_i > A_i$)
\[
  \vvv{v} = 100\,\uvec{\i} + 200\,\uvec{\j}\,\unit{\centi\metre}
\]
las nuevas componentes serían $(100, 200)\,\unit{\centi\metre}$.
En otras palabras, la base ``se ha hecho 100 veces menor'', pero las componentes
``se han hecho 100 veces mayor''.

\subsubsection{Concepto}
Estas componentes han variado de forma contraria a como lo hizo la base elegida.
A las componentes que se comportan de esta manera se les llama \emph{contravariantes} y,
aunque no es necesario en la mayor parte de las situaciones, se suele utilizar el índice
que las acompaña como un \emph{superíndice} (que no indican ninguna potencia o
exponente). Así, un vector como el que hemos puesto de ejemplo, se podría representar en
forma genérica como
\[
  \vvv{v} = A^1\vvv{b}_1 + A^2\vvv{b}_2
\]
En notación de Einstein sería
\[
  \vvv{v} = A^i\vvv{b}_i
\]

\subsubsection{Representación gráfica de las componentes contravariantes}
La componentes contravariantes en el sistema de coordenadas oblícuas se pueden
representar gráficamente, ver figura \ref{fig:R2-oblic_contrav}.
\begin{figure}[ht]
  % Escala
  \def\scl{1.00}
  % 
  \pgfmathsetmacro{\EJESCARTLONG}{3.0}
  \pgfmathsetmacro{\EJEOBLICLONG}{2.7}
  % Eje x
  %\pgfmathsetmacro{\XLONG}{\EJESCARTLONG}
  % Eje y
  %\pgfmathsetmacro{\YLONG}{\EJESCARTLONG}
  % Eje xtilde
  \pgfmathsetmacro{\XTILDEANG}{0}
  \pgfmathsetmacro{\XTILDELONG}{\EJEOBLICLONG}    
  % Eje ytilde
  \pgfmathsetmacro{\YTILDEANG}{70}
  \pgfmathsetmacro{\YTILDELONG}{\EJESCARTLONG}
  % Vector P
  \pgfmathsetmacro{\PMOD}{2.3}
  \pgfmathsetmacro{\PANG}{40}
  % Fondo
  \pgfmathsetmacro{\HORZ}{0.25}
  \pgfmathsetmacro{\VERT}{0.25}
  % 
  \centering
  \begin{tikzpicture}[%
    scale=\scl,
    every node/.style={black,font=\small},
    %ejecart/.style={->, black!30},
    ejeoblic/.style={->, black},
    proyeccion/.style={black!20, line width=0.4},
    % longitud/.style={line width=.5pt,\colorV},
    texto/.style={black},
    %textoapagado/.style={\colorTorig},
    vector/.style={-{Latex[round]}, shorten >=1.2pt, line width=.8pt,\colorV},
    contravariante/.style={%
      -{Latex[round,width=5pt,length=8pt]},line width=1.2pt, orange!80!black},
    contravtxt/.style={orange!60!black},
    pcirculo/.style={fill=\colorPorig, draw=black},
    background/.style={%
      line width=\bgborderwidth,
      draw=\bgbordercolor,
      fill=\bgcolor,
    },
    ]
    % COORDENADAS
    % Origen
    \coordinate (O) at (0,0);
    % Eje xtilde
    \path[save path=\ejextilde] (O) -- ++(\XTILDEANG:\XTILDELONG cm)
    coordinate (xtildefin);
    \node[texto, below] at (xtildefin) {\footnotesize $\tilde{x}^1$};
    % Eje ytilde
    \path[save path=\ejeytilde, name path=O--ytilde] (O) -- ++(\YTILDEANG:\YTILDELONG cm)
    coordinate (ytildefin);
    % Vector
    \path[save path=\r] (O) -- coordinate[pos=0.6] (posvecttxt) (\PANG:\PMOD cm)
    coordinate (P);
    % Proyección de P en los ejes
    % Con eje x
    \path[save path=\proyx] (e2fin) |- (xfin) coordinate[pos=0.5] (e2x) (xfin);
    % Componentes contravariantes
    % Intersección de horizontal desde P hasta ytilde
    % -> Componente contravariante eje ytilde (CTV2) 
    \path[name path=P--y] (P) -| (yfin);
    \path[name intersections={of=P--y and O--ytilde,by=CTV2}];
    % Paralela (CTV2) -- (O) que pase por (P)
    % -> Componente contravariante eje xtilde (CTV1)
    \path (P) -- ($(P) - (CTV2) - (O)$) coordinate (CTV1);
    
    % DIBUJOS
    % Proyecciones de componentes
    % Contravariantes
    \draw[proyeccion] (P) -- (CTV1);
    %\node[below] at (CTV1) {\footnotesize $\hat{x}^1$};
    \draw[proyeccion] (P) -- (CTV2);
    %\node[left] at (CTV2) {\footnotesize $\hat{x}^2$};
    % xtilde
    \draw[ejeoblic, use path=\ejextilde];
    % ytilde
    \draw[ejeoblic, use path=\ejeytilde];
    \node[above right=0pt and -4pt, texto] at (ytildefin) {$\tilde{x}^2$};
    % Vector P
    \draw[vector, use path=\r];
    \node[above left=-3pt and -1pt] at (posvecttxt) {\footnotesize $\vvv{v}$};
    \filldraw[fill=green, draw=black] (P) circle[radius=1.2pt];
    \node[above right=0pt and -5pt] at (P) {\footnotesize $P$};

    % Marcar en color la componentes contravariantes
    \draw[contravariante]
    (O) -- node[contravtxt,pos=0.5, below]
    {$\hat{A}^1\xhat{e}_1$} (CTV1);
    \draw[contravariante] (O) -- node[contravtxt,pos=0.5,above left=-2pt and -2pt]
    {$\hat{A}^2\xhat{e}_2$} (CTV2);
    
    % Origen
    \fill[orange!80!black] (O) circle [radius=1pt];
    
    % Fondo amarillo
    \coordinate (SW) at ($(current bounding box.south west) + (-\HORZ cm,-\VERT cm)$);
    \coordinate (NE) at ($(current bounding box.north east) + (\HORZ cm,\VERT cm)$);
    \begin{scope}[on background layer]
      \draw[background] (SW) rectangle (NE);
    \end{scope}
  \end{tikzpicture}
  \caption{Componentes contravariantes $A^i$ en coordenadas oblícuas. El vector $\vvv{v}$
    se representa como $\vvv{v}=A^1\,\xhat{e}_1 + A^2\,\xhat{e}_2$. Las componentes
    contravariantes nos indican el desplazamiento en el espacio.}
  \label{fig:R2-oblic_contrav}
\end{figure}

Para centrar ideas, supongamos que el vector $\vvv{v}$ de la figura representa un
desplazamiento en $\symbb{R}^2$. la suma de Las componentes contravariantes $A^i$
multiplicadas por sus respectivos vectores unidad, nos da el desplazamiento que nos señala el vector.


\subsection{Forma matricial de la base y de las componentes}
Recordemos que los vectores de una base se pueden representar matricialmente situando los
vectores en fila
\[
  B =
  \begin{pmatrix}
    \vvv{b}_1 & \vvv{b}_2
  \end{pmatrix}
\]

En forma matricial, las componentes contravariantes del vector $\vvv{v}$, en una cierta
base, se representan en columna. Así, el vector anterior se puede expresar en la base
elegida como
\[
  \vvv{v} =
  \begin{pmatrix}
    A^1\\
    A^2
  \end{pmatrix}
\]
Es importante distinguir entre estas dos notaciones y darse cuenta de que no son
comparables. La primera es la forma en la que se representa un vector en función de sus
componentes contravariantes en una cierta base, mientras que la segunda es la forma en
la que se representa un conjunto de vectores que forman una base. Lo único que tienen en
común es que son una representación matricial, aunque de cosas diferentes.


\subsection{Transformación de componentes}
Las componentes contravariantes de un vector se representan en columna y en la expresión
matemática interviene la matriz inversa (las componentes contravariantes se transforman
de forma inversa a como lo hacían los vectores de la base). Así, las nuevas componentes se
obtienen a partir de las antiguas mediante $\mmm{M}^{-1}$, mientras que las antiguas se
transforman a partir de las nuevas mediante $\mmm{M}$, la inversa de la anterior
\begin{subequations}
  \begin{align}
    &\left[A^i\right]_{B'} = \mmm{M}^{-1} \left[A^i\right]_{B}\\
    &\left[A^i\right]_{B}\hspace{0.4em} = \mmm{M} \left[A^i\right]_{B'}
  \end{align}
\end{subequations}

En coordenadas oblícuas en $\symbb{R}^2$
\begin{subequations}
  \begin{align}
    \begin{pmatrix}
      \tilde{A}^1\\
      \tilde{A}^2
    \end{pmatrix}
    &=
      \begin{pmatrix}
        1 & -\cot\alpha\\
        0 & \phantom{-}\csc\alpha
      \end{pmatrix}
      \,
    \begin{pmatrix}
      A^1\\
      A^2
    \end{pmatrix}\\
    \begin{pmatrix}
      A^1\\
      A^2
    \end{pmatrix}
    &=
      \begin{pmatrix}
        1 & \cos\alpha\\
        0 & \sin\alpha
      \end{pmatrix}
      \,
    \begin{pmatrix}
      \tilde{A}^1\\
      \tilde{A}^2
    \end{pmatrix}
  \end{align}
\end{subequations}


\subsubsection{Ejemplo de transformación de componentes}
Supongamos que tenemos un vector en coordenadas rectangulares
\[
  \vvv{v}
  = A^1\,\xhat{u}_1 + A^2\,\xhat{u}_1 = A^i\,\xhat{u}_i
  = 2\,\uvec{\i} - 4\,\uvec{\j}
\]
y estamos interesados en obtener las componentes $\tilde{A}^i$ en un sistema de
coordenadas oblícuas, donde el ángulo que forman sus ejes vale
$\alpha = \pi/3\,\unit{\radian}$. Tendremos que aplicar la inversa de la matriz de
transformación de base de $B$ a $B'$
\begin{align*}
  \begin{pmatrix}
    \tilde{A}^1\\
    \tilde{A}^2
  \end{pmatrix}
  &=
  \begin{pmatrix}
    1 & -\cot{\pi/3}\\
    0 & \phantom{-}\csc{\pi/3}
  \end{pmatrix}
  \,
  \begin{pmatrix}
    A^1\\
    A^2
  \end{pmatrix}
    =
    \begin{pmatrix}
      1 & -1/\sqrt{3}\\
      0 & \phantom{-}2/\sqrt{3}
    \end{pmatrix}
    \,
    \begin{pmatrix}
      \phantom{-}2\\
      -4
    \end{pmatrix}
    =
    \begin{pmatrix}
      2+\dfrac{4}{\sqrt{3}}\\[3ex]
      -\dfrac{8}{\sqrt{3}}
    \end{pmatrix}
\end{align*}

Así
\[
  \vvv{v}
  = \tilde{A}^1\,\xhat{e}_1 + \tilde{A}^2\,\xhat{e}_2 = \tilde{A}^i\,\xhat{e}_i
  = \left(2+\dfrac{4}{\sqrt{3}}\right)\,\xhat{e}_1 - \dfrac{8}{\sqrt{3}}\,\xhat{e}_2
\]


Si nos dieran las componentes en coordenadas oblícuas
\[
  \left(\tilde{A}^1, \tilde{A}^2\right) = \left(2+4/\sqrt{3}, -8/\sqrt{3}\right)
\]
y quisiéramos obtener las coordenadas rectangulares, deberíamos obtener $(2,-4)$
\begin{align*}
  \begin{pmatrix}
    A^1\\
    A^2
  \end{pmatrix}
  &=
  \begin{pmatrix}
    1 & \cos{\pi/3}\\
    0 & \sin{\pi/3}
  \end{pmatrix}
  \,
  \begin{pmatrix}
    \tilde{A}^1\\
    \tilde{A}^2
  \end{pmatrix}
    =
  \begin{pmatrix}
    1 & 1/2\\
    0 & \sqrt{3}/2
  \end{pmatrix}
  \,
  \begin{pmatrix}
    2+4/\sqrt{3}\\
    -8/\sqrt{3}
  \end{pmatrix}\\
  &=
    \begin{pmatrix}
      2+4/\sqrt{3}-4/\sqrt{3}\\
      -\sqrt{3}/2\, \left(-8/\sqrt{3}\right)
    \end{pmatrix}
    =
    \begin{pmatrix}
      \phantom{-}2\\
      -4
    \end{pmatrix}  
\end{align*}


\subsection{Métrica}
Con los conocimientos que hemos adquirido, la forma más rápida para obtener la métrica
en coordenadas oblícuas es utilizar el método seguido en la expresión
\eqref{eq:R2-cart-metrica}. Para ello debemos calcular el producto escalar de los
vectores de la base \eqref{eq:R2-oblic_base}.

Multiplicamos escalarmente estos vectores
\begin{subequations}
  \begin{align}
    \label{eq:R2-oblic-e1e1_prod_escalar}
    \xhat{e}_1\cdot\xhat{e}_1
    &=
      \uvec{\i}\cdot\uvec{\i} = 1\\
    \label{eq:R2-oblic-e1e2_prod_escalar}    
    \xhat{e}_1\cdot\xhat{e}_2
    &=
      \uvec{\i}\cdot (\cos\theta\,\uvec{i} + \sin\theta\,\uvec{\j}) = \cos\theta\\
    \label{eq:R2-oblic-e2e1_prod_escalar}
    \xhat{e}_2\cdot\xhat{e}_1
    &=
      \xhat{e}_1\cdot\xhat{e}_2 = \cos\theta\\
    \label{eq:R2-oblic-e2e2_prod_escalar}
    \xhat{e}_2\cdot\xhat{e}_2
    &=
      (\cos\theta\,\uvec{\i}+\sin\theta\,\uvec{j})
      \cdot (\cos\theta\,\uvec{\i}+\sin\theta\,\uvec{j})
      = (\cos^2\theta + \sin^2\theta) = 1
  \end{align}
\end{subequations}

La métrica en coordenadas oblícuas queda
\begin{equation}\label{eq:R2-oblic_metrica}
  \mmm{g}
  = g_{ij}
  = \begin{pmatrix}
    g_{11} & g_{12}\\
    g_{21} & g_{22}
    \end{pmatrix}
  = \begin{pmatrix}
      \xhat{e}_1\cdot\xhat{e}_1 & \xhat{e}_1\cdot\xhat{e}_2\\
      \xhat{e}_2\cdot\xhat{e}_1 & \xhat{e}_2\cdot\xhat{e}_2
  \end{pmatrix}
  = \begin{pmatrix}
    1 & \cos\alpha\\
    \cos\alpha & 1
  \end{pmatrix}
\end{equation}


\subsection{Componentes covariantes}
Vamos a suponer que un vector $\vvv{v}$ representa un gradiente de un campo escalar
$\vvv{\nabla}\phi$.
Representaremos las componentes covariantes con subíndice $A_i$.
\begin{figure}[ht]
  % Escala
  \def\scl{1.00}
  % 
  \pgfmathsetmacro{\EJESCARTLONG}{3.0}
  \pgfmathsetmacro{\EJEOBLICLONG}{2.7}
  % Eje xtilde
  \pgfmathsetmacro{\XTILDEANG}{0}
  \pgfmathsetmacro{\XTILDELONG}{\EJEOBLICLONG}    
  % Eje ytilde
  \pgfmathsetmacro{\YTILDEANG}{70}
  \pgfmathsetmacro{\YTILDELONG}{\EJESCARTLONG}
  % Vector P
  \pgfmathsetmacro{\PMOD}{2.3}
  \pgfmathsetmacro{\PANG}{40}
  % Fondo
  \pgfmathsetmacro{\HORZ}{0.25}
  \pgfmathsetmacro{\VERT}{0.25}
  % 
  \centering
  \begin{tikzpicture}[%
    scale=\scl,
    every node/.style={black,font=\small},
    ejeoblic/.style={->, black},
    proyeccion/.style={black!20, line width=0.4},
    texto/.style={black},
    vector/.style={-{Latex[round]}, shorten >=1.2pt, line width=.8pt,\colorV},
    covariante/.style={%
      -{Latex[round,width=5pt,length=8pt]},line width=1.2pt, cyan!70!black},
    covtxt/.style={cyan!40!black},
    pcirculo/.style={fill=\colorPorig, draw=black},
    background/.style={%
      line width=\bgborderwidth,
      draw=\bgbordercolor,
      fill=\bgcolor,
    },
    ]
    % COORDENADAS
    % Origen
    \coordinate (O) at (0,0);
    % Eje xtilde
    \path[save path=\ejextilde] (O) -- ++(\XTILDEANG:\XTILDELONG cm)
    coordinate (xtildefin);
    % Eje ytilde
    \path[save path=\ejeytilde, name path=O--ytilde] (O) -- ++(\YTILDEANG:\YTILDELONG cm)
    coordinate (ytildefin);
    % Vector
    \path[save path=\r] (O) -- coordinate[pos=0.6] (posvecttxt) (\PANG:\PMOD cm)
    coordinate (P);
    % Componentes covariantes
    % Proyección de P en los ejes
    % Con eje x
    \path[save path=\proyx] (e2fin) |- (xfin) coordinate[pos=0.5] (e2x) (xfin);
    % Componente covariante con eje xtilde (COV1)
    \path (P) |- coordinate[pos=0.5] (COV1) (xtildefin);
    % Perpendicular a O--ytilde que pasa por P
    % -> Componente covariante con eje ytilde (COV2)
    \path (P) -- ($(O)!(P)!(ytildefin)$) coordinate (COV2);
 
    
    % DIBUJOS
    % Covariantes
    \draw[proyeccion] (P) -- (COV1);
    \draw[proyeccion] (P) -- (COV2);
    
    % Ejes
    % xtilde
    \draw[ejeoblic, use path=\ejextilde];
    \node[texto, right] at (xtildefin) {\footnotesize $\tilde{x}^1$};
    % ytilde
    \draw[ejeoblic, use path=\ejeytilde];
    \node[above right=0pt and -4pt, texto] at (ytildefin) {$\tilde{x}^2$};
    % \node[rotate=90, left=8pt] at (ytexto) {$r\sin\theta$};
    % Vector P
    \draw[vector, use path=\r];
    \node[above left=-3pt and -1pt] at (posvecttxt) {\footnotesize $\vvv{v}$};
    \filldraw[fill=green, draw=black] (P) circle[radius=1.2pt];
    \node[above right=0pt and -5pt] at (P) {\footnotesize $P$};

    % Marcar en gris la componentes contravariantes
    \draw[covariante]
    (O) -- node[covtxt,pos=0.5, below]
    {$\hat{A}_1\xhat{e}^1$} (COV1);
    \draw[covariante] (O) -- node[covtxt,pos=0.5,above left=-2pt and -2pt]
    {$\hat{A}_2\xhat{e}^2$} (COV2);
    
    % Origen
    \fill[orange!80!black] (O) circle [radius=1pt];
    
    % Fondo amarillo
    \coordinate (SW) at ($(current bounding box.south west) + (-\HORZ cm,-\VERT cm)$);
    \coordinate (NE) at ($(current bounding box.north east) + (\HORZ cm,\VERT cm)$);
    \begin{scope}[on background layer]
      \draw[background] (SW) rectangle (NE);
    \end{scope}
  \end{tikzpicture}
  \caption{Componentes covariantes $A_i$ en coordenadas oblícuas. El vector $\vvv{v}$
    se representa como $\vvv{v}=A_1\,\xhat{e}^1 + A_2\,\xhat{e}^2$. Las componentes
    contravariantes nos indican el desplazamiento en el espacio.}
  \label{fig:R2-oblic_cov}
\end{figure}













\begin{figure}[ht]
  % Escala
  \def\scl{1.00}
  % 
  \pgfmathsetmacro{\EJESCARTLONG}{3.0}
  \pgfmathsetmacro{\EJEOBLICLONG}{2.7}
  % Eje x
  \pgfmathsetmacro{\XLONG}{\EJESCARTLONG}
  % Eje y
  \pgfmathsetmacro{\YLONG}{\EJESCARTLONG}
  % Eje xtilde
  \pgfmathsetmacro{\XTILDEANG}{0}
  \pgfmathsetmacro{\XTILDELONG}{\EJEOBLICLONG}    
  % Eje ytilde
  \pgfmathsetmacro{\YTILDEANG}{70}
  \pgfmathsetmacro{\YTILDELONG}{\EJESCARTLONG}
  % Vector P
  \pgfmathsetmacro{\PMOD}{2.3}
  \pgfmathsetmacro{\PANG}{40}
  % Fondo
  \pgfmathsetmacro{\HORZ}{0.25}
  \pgfmathsetmacro{\VERT}{0.25}
  % 
  \centering
  \begin{tikzpicture}[%
    scale=\scl,
    every node/.style={black,font=\small},
    ejecart/.style={->, black!30},
    ejeoblic/.style={->, black},
    proyeccion/.style={black!20, line width=0.4},
    % longitud/.style={line width=.5pt,\colorV},
    texto/.style={black},
    textoapagado/.style={\colorTorig},
    vector/.style={-{Latex[round]}, shorten >=1.2pt, line width=.8pt,\colorV},
    %vtilde/.style={%
    %  -{Latex[round,width=4pt,length=6pt]}, line width=1pt, \colorVred},
    pcirculo/.style={fill=\colorPorig, draw=black},
    angulo/.style={fill=green!15, draw=\colorGirodos},
    background/.style={%
      line width=\bgborderwidth,
      draw=\bgbordercolor,
      fill=\bgcolor,
    },
    ]
    % COORDENADAS
    % Origen
    \coordinate (O) at (0,0);
    % Eje x
    \path[save path=\ejex] (O) -- (\XLONG cm, 0) coordinate (xfin);
    % Eje y
    \path[save path=\ejey] (O) -- (0, \YLONG cm) coordinate (yfin);
    % Eje xtilde
    \path[save path=\ejextilde] (O) -- ++(\XTILDEANG:\XTILDELONG cm)
    coordinate (xtildefin);
    \node[texto, below] at (xtildefin) {\footnotesize $\tilde{x}^1$};
    % Eje ytilde
    \path[save path=\ejeytilde, name path=O--ytilde] (O) -- ++(\YTILDEANG:\YTILDELONG cm)
    coordinate (ytildefin);
    % Vector
    \path[save path=\r] (O) -- coordinate[pos=0.5] (pmid) (\PANG:\PMOD cm)
    coordinate (P);
    % Proyección de P en los ejes
    % Con eje x
    \path[save path=\proyx] (e2fin) |- (xfin) coordinate[pos=0.5] (e2x) (xfin);
    % Posición de texto en eje x
    % \path (O) -- coordinate[pos=0.5] (extxt) (e2x);
    % Con eje y
    \path[save path=\proyy] (e2fin) -| coordinate[pos=0.5] (e2y) (yfin);
    % Posición de texto en eje y
    % \path (O) -- coordinate[pos=0.5] (eytxt) (e2y);
    % Componentes covariantes y contravariantes
    % Intersección de horizontal desde P hasta ytilde
    % -> Componente contravariante eje ytilde (CTV2) 
    \path[name path=P--y] (P) -| (yfin);
    \path[name intersections={of=P--y and O--ytilde,by=CTV2}];
    % Paralela (CTV2) -- (O) que pase por (P)
    % -> Componente contravariante eje xtilde (CTV1)
    \path (P) -- ($(P) - (CTV2) - (O)$) coordinate (CTV1);
    % Componente covariante con eje xtilde (COV1)
    \path (P) |- coordinate[pos=0.5] (COV1) (xtildefin);
    % Perpendicular a O--ytilde que pasa por P
    % -> Componente covariante con eje ytilde (COV2)
    \path (P) -- ($(O)!(P)!(ytildefin)$) coordinate (COV2);
    % Proyecciones de las componentes
    
    % DIBUJOS
    % Ángulo theta
    \path (xfin) -- (O) -- (ytildefin) pic
    [angulo, angle radius=15pt]
    {angle = xfin--O--ytildefin};
    \node[above right=-2pt and 4pt] at (O) {\scriptsize $\alpha$};
%    
    % Proyecciones de componentes
    % Contravariantes
    \draw[proyeccion] (P) -- (CTV1);
    \node[below] at (CTV1) {\footnotesize $\hat{x}^1$};
    \draw[proyeccion] (P) -- (CTV2);
    \node[left] at (CTV2) {\footnotesize $\hat{x}^2$};
    % Covariantes
    \draw[proyeccion] (P) -- (COV1);
    \node[below] at (COV1) {\footnotesize $\hat{x}_1$};
    \draw[proyeccion] (P) -- (COV2);
    \node[above left=-2pt and -2pt] at (COV2) {\footnotesize $\hat{x}_2$};
    
    % Ejes
    % x
    \draw[ejecart, use path=\ejex];
    \node[textoapagado, below=3pt] at (xfin) {\small $x$};
    % y
    \draw[ejecart, use path=\ejey];
    \node[above, textoapagado] at (yfin) {$y$};
    % xtilde
    \draw[ejeoblic, use path=\ejextilde];
    % \node[below, texto] at (xtildefin) {$\tilde{x}^1$};
    % ytilde
    \draw[ejeoblic, use path=\ejeytilde];
    \node[above right=0pt and -4pt, texto] at (ytildefin) {$\tilde{x}^2$};
    % \node[rotate=90, left=8pt] at (ytexto) {$r\sin\theta$};
    % Vector P
    \draw[vector, use path=\r];
    \node[above left=-3pt and -1pt] at (pmid) {\footnotesize $\vvv{r}$};
    \filldraw[fill=green, draw=black] (P) circle[radius=1.2pt];
    \node[above right=0pt and -5pt] at (P) {\footnotesize $P$};
    
%    % Versores
%    \draw[vtilde, use path=\vxtilde];
%    \node[\colorTred, below] at (vxtildetxt) {\footnotesize $\xhat{e}_1$};
%    \draw[vtilde, use path=\vytilde];
%    \node[\colorTred,above left=-4pt and 2pt] at (vytildetxt)
%    {\footnotesize $\xhat{e}_2$};
%    %      % Coordenada r y punto  P
%    % \draw[longitud, use path=\r];
%    % \node[above left=-3pt and -1pt] at (pmid) {$r$};
%    % \filldraw[fill=green, draw=black] (P) circle[radius=1.2pt];
%    % \node[above] at (P) {$P$};
    
    % Origen
    \filldraw (O) circle [radius=.2pt];
    
    % Fondo amarillo
    \coordinate (SW) at ($(current bounding box.south west) + (-\HORZ cm,-\VERT cm)$);
    \coordinate (NE) at ($(current bounding box.north east) + (\HORZ cm,\VERT cm)$);
    \begin{scope}[on background layer]
      \draw[background] (SW) rectangle (NE);
    \end{scope}
  \end{tikzpicture}
  \caption{Componentes contravariantes y covariantes. El ángulo $\alpha$ entre los ejes
  está marcado en verde.}
\end{figure}









%
%\section{Coordenadas polares planas}
%Aquí se ponen las cosas algo más interesantes. En este sistema, se utilizan dos
%coordenadas, $r$ y $\theta$. La primera coordenada es la distancia desde el punto del espacio al origen de coordenadas, y la segunda es el ángulo que forma el eje de abcisas
%con la línea que une el origen de coordenadas con el punto, ver figura
%\ref{fig:R2-coord_polares}.
%
%\begin{figure}[ht]
%  \centering
%  \begin{minipage}{0.40\linewidth}
%    % Escala
%    \def\scl{1.25}
%    % 
%    \pgfmathsetmacro{\EJESEXTRA}{-0.2}
%    \pgfmathsetmacro{\EJESLONG}{2.3}
%    % Eje x
%    \pgfmathsetmacro{\XMLONG}{\EJESEXTRA}
%    \pgfmathsetmacro{\XPLONG}{\EJESLONG}
%    % Eje y
%    \pgfmathsetmacro{\YMLONG}{\EJESEXTRA}
%    \pgfmathsetmacro{\YPLONG}{\EJESLONG}
%    % Vector P
%    \pgfmathsetmacro{\PMOD}{2.5}
%    \pgfmathsetmacro{\PANG}{40}
%    % Fondo
%    \pgfmathsetmacro{\HORZ}{0.25}
%    \pgfmathsetmacro{\VERT}{0.25}
%    % 
%    \centering
%    \begin{tikzpicture}[%
%      scale=\scl,
%      every node/.style={black,font=\small},
%      eje/.style={->},
%      proyeccion/.style={black!20, line width=0.4},
%      longitud/.style={line width=.5pt,\colorV},
%      textooriginal/.style={\colorTorig},
%      pcirculo/.style={fill=\colorPorig, draw=black},
%      angulogirado/.style={fill=\colorAngGreen, draw=\colorGirodos},
%      background/.style={%
%        line width=\bgborderwidth,
%        draw=\bgbordercolor,
%        fill=\bgcolor,
%      },
%      ]
%      % COORDENADAS
%      % Origen
%      \coordinate (O) at (0,0);
%      % Extremo izdo del eje x e inferior del eje y
%      \coordinate (xini) at (\XMLONG cm,0);
%      \coordinate (yini) at (0,\YMLONG cm);
%      % Eje x
%      \path[save path=\ejex] (xini) -- coordinate[pos=0.5] (xtexto) (\XPLONG cm, 0)
%      coordinate (xfin);
%      % Eje y
%      \path[save path=\ejey] (yini) -- coordinate[pos=0.6] (ytexto) (0, \YPLONG cm)
%      coordinate (yfin);
%      % Vector
%      \path[save path=\r] (O) -- coordinate[pos=0.5] (pmid) (\PANG:\PMOD cm)
%      coordinate (P);
%      % Proyección de P en los ejes
%      \path[save path=\proyx] (P) |- (xfin);
%      \path[save path=\proyy] (P) -| (yfin);
%
%      % DIBUJOS
%      % Ángulo theta
%      \path (xfin) -- (O) -- (P) pic
%      [-{Latex[width=2.5pt,length=3.7pt]},angulogirado,
%      "\footnotesize $\theta$",angle radius=20pt, angle eccentricity=0.68]
%      {angle = xfin--O--P};
%      % Proyección de P en los ejes
%      \draw[proyeccion, use path=\proyx];
%      \draw[proyeccion, use path=\proyy];
%      % Ejes
%      % x
%      \draw[eje, use path=\ejex];
%      \node[right] at (xfin) {$x$};
%      %\node[below] at (xtexto) {$r\cos\theta$};
%      % y
%      \draw[eje, use path=\ejey];
%      \node[above] at (yfin) {$y$};
%      %\node[rotate=90, left=8pt] at (ytexto) {$r\sin\theta$};
%      % Coordenada r y punto  P
%      \draw[longitud, use path=\r];
%      \node[above left=-3pt and -1pt] at (pmid) {$r$};
%      \filldraw[fill=green, draw=black] (P) circle[radius=1.2pt];
%      \node[above] at (P) {$P$};
%
%      % Origen
%      \filldraw (O) circle [radius=.2pt];
%      % Fondo amarillo
%      \coordinate (SW) at ($(current bounding box.south west) + (-\HORZ cm,-\VERT cm)$);
%      \coordinate (NE) at ($(current bounding box.north east) + (\HORZ cm,\VERT cm)$);
%      \begin{scope}[on background layer]
%        \draw[background] (SW) rectangle (NE);
%      \end{scope}
%    \end{tikzpicture}
%  \caption{Coordenadas polares $r$ y $\theta$ de un punto $P$ del plano.}
%  \label{fig:R2-coord_polares}
%\end{minipage}
%\hspace{1em}
%\begin{minipage}{0.40\linewidth}
%    % Escala
%    \def\scl{1.20}
%    % 
%    \pgfmathsetmacro{\EJESEXTRA}{-0.2}
%    \pgfmathsetmacro{\EJESLONG}{2.3}
%    % Eje x
%    \pgfmathsetmacro{\XMLONG}{\EJESEXTRA}
%    \pgfmathsetmacro{\XPLONG}{\EJESLONG}
%    % Eje y
%    \pgfmathsetmacro{\YMLONG}{\EJESEXTRA}
%    \pgfmathsetmacro{\YPLONG}{\EJESLONG}
%    % Vector P
%    \pgfmathsetmacro{\PMOD}{2.5}
%    \pgfmathsetmacro{\PANG}{40}
%    % Fondo
%    \pgfmathsetmacro{\HORZ}{0.25}
%    \pgfmathsetmacro{\VERT}{0.25}
%    % 
%    \centering
%    \begin{tikzpicture}[%
%      scale=\scl,
%      every node/.style={black,font=\small},
%      eje/.style={->},
%      proyeccion/.style={black!20, line width=0.4},
%      vector/.style={-{Latex[round]}, shorten >=1.2pt, line width=.8pt,\colorV},
%      textooriginal/.style={\colorTorig},
%      pcirculo/.style={fill=\colorPorig, draw=black},
%      angulogirado/.style={fill=\colorAngGreen, draw=\colorGirodos},
%      background/.style={%
%        line width=\bgborderwidth,
%        draw=\bgbordercolor,
%        fill=\bgcolor,
%      },
%      ]
%      % COORDENADAS
%      % Origen
%      \coordinate (O) at (0,0);
%      % Extremo izdo del eje x e inferior del eje y
%      \coordinate (xini) at (\XMLONG cm,0);
%      \coordinate (yini) at (0,\YMLONG cm);
%      % Eje x
%      \path[save path=\ejex] (xini) -- coordinate[pos=0.5] (xtexto) (\XPLONG cm, 0)
%      coordinate (xfin);
%      % Eje y
%      \path[save path=\ejey] (yini) -- coordinate[pos=0.6] (ytexto) (0, \YPLONG cm)
%      coordinate (yfin);
%      % Vector
%      \path[save path=\vector] (O) -- coordinate[pos=0.5] (pmid) (\PANG:\PMOD cm)
%      coordinate (P);
%      % Proyección de P en los ejes
%      \path[save path=\proyx] (P) |- (xfin);
%      \path[save path=\proyy] (P) -| (yfin);
%
%      % DIBUJOS
%      % Ángulo theta
%      \path (xfin) -- (O) -- (P) pic
%      [-{Latex[width=2.5pt,length=3.7pt]},angulogirado,
%      "\footnotesize $\theta$",angle radius=20pt, angle eccentricity=0.68]
%      {angle = xfin--O--P};
%      % Proyección de P en los ejes
%      \draw[proyeccion, use path=\proyx];
%      \draw[proyeccion, use path=\proyy];
%      % Ejes
%      % x
%      \draw[eje, use path=\ejex];
%      \node[right] at (xfin) {$x$};
%      \node[below] at (xtexto) {$r\cos\theta$};
%      % y
%      \draw[eje, use path=\ejey];
%      \node[above] at (yfin) {$y$};
%      \node[rotate=90, left=8pt] at (ytexto) {$r\sin\theta$};
%      % Vector y punto  P
%      \draw[vector, use path=\vector];
%      \node[above left=-3pt and -1pt] at (pmid) {$\vvv{r}$};
%      \filldraw[fill=green, draw=black] (P) circle[radius=1.2pt];
%      \node[above] at (P) {$P$};
%
%      % Origen
%      \filldraw (O) circle [radius=.2pt];
%      % Fondo amarillo
%      \coordinate (SW) at ($(current bounding box.south west) + (-\HORZ cm,-\VERT cm)$);
%      \coordinate (NE) at ($(current bounding box.north east) + (\HORZ cm,\VERT cm)$);
%      \begin{scope}[on background layer]
%        \draw[background] (SW) rectangle (NE);
%      \end{scope}
%    \end{tikzpicture}
%    \caption{Vector de posición $\vvv{r}$, y coordenadas $x$ e $y$ en función de $r$ y
%      $\theta$.}
%  \label{fig:R2:vector-r-R2}
%\end{minipage}
%\end{figure}
%
%\subsection{Relación con las coordenadas cartesianas}
%De la figura \ref{fig:R2:vector-r-R2} se desprenden las siguientes relaciones
%\begin{subequations}
%  \begin{align}\label{eq:R2-x-rtheta}
%    x &= r\cos\theta\\
%    \label{eq:R2-y-rtheta}
%    y &= r\sin\theta
%  \end{align}
%\end{subequations}
%y sus inversas
%\begin{subequations}
%  \begin{align}\label{eq:R2-r-xy}
%    r &= \sqrt{x^2 + y^2}\\
%    \label{eq:R2-theta-xy}
%    \theta &= \arctan\left(\frac{y}{x}\right)
%  \end{align}
%\end{subequations}
%Nótese que el ángulo $\theta$ no está definido en el origen. Por eso en este sistema el
%origen es un polo (las coordenadas no están completamente definidas).
%
%\subsection{Bases, vectores unitarios, sus productos escalares y derivadas}
%El vector $\vvv{r}$ que representa cualquier punto de $\symbb{R}^2$ se puede escribir, en
%la base natural de coordenadas cartesianas, como
%\begin{equation}
%  \vvv{r} = r\cos\theta\,\uvec{\i} + r\sin\theta\,\uvec{\j}
%\end{equation}
%\subsubsection{Derivadas parciales de la posición}
%La base natural en polares está formada por los vectores
%$B = \{\vvv{e}_r, \vvv{e}_{\theta}\}$ ---no necesariamente unitarios--- que llevan
%información del cambio de posición al variar cada coordenada forman una base
%$\set{\vvv{e}_r, \vvv{e}_\theta}$ y se calculan derivando parcialmente el vector
%$\vvv{r}$ con respecto de las variables $r$ y $\theta$
%\begin{subequations}
%\begin{align}
%  \vvv{e}_r
%  &=
%    \dfrac{\partial\vvv{r}}{\partial r}
%    = \cos\theta\,\uvec{\i} + \sin\theta\,\uvec{\j}
%    \longrightarrow
%    \|\vvv{e}_r\| = 1\\
%  \vvv{e}_\theta
%  &=
%    \dfrac{\partial\vvv{r}}{\partial\theta}
%    = -r\sin\theta\,\uvec{\i} + r\cos\theta\,\uvec{\j}
%    \longrightarrow
%    \|\vvv{e}_\theta\| = r\\    
%\end{align}
%\end{subequations}
%Estos vectores se utilizan en geometría diferencial y son los que se utilizarán para
%calcular la métrica en coordenadas polares. Nótese que, en particular, $\vvv{e}_\theta$ no
%es un vector unitario, su módulo vale $r$.
%
%\subsubsection{Vectores unitarios}
%En física estamos interesados en bases unitarias. La base unitaria en polares la
%representaremos como $B=\set{\xhat{r}, \xhat{\theta}}$.
%Calculamos los vectores unitarios en polares en función de la base cartesiana
%{\small
%\begin{subequations}
%  \begin{align}\label{eq:R2-versor-r}
%    \xhat{r}
%    &= \dfrac{\vvv{e}_r}{|\vvv{e}_r|}
%    =\dfrac{\cos\theta\,\uvec{\i}+\sin\theta\,\uvec{\j}}
%    {\sqrt{\cos^2\theta+\sin^2\theta}}
%    = \dfrac{\cos\theta\,\uvec{\i}+\sin\theta\,\uvec{\j}}{1}
%      = \cos\theta\,\uvec{\i} + \sin\theta\,\uvec{\j}\\
%    \label{eq:R2-versor-theta}
%  \xhat{\theta}
%  &= \dfrac{\vvv{e}_\theta}{|\vvv{e}_\theta|}
%  = \dfrac{-r\sin\theta\,\uvec{\i}+\cos\theta\,\uvec{\j}}
%  {\sqrt{r^2(\cos^2\theta+\sin^2\theta)}}
%  = \dfrac{-r\sin\theta\,\uvec{\i}+r\cos\theta\,\uvec{\j}}{r}
%  = -\sin\theta\,\uvec{\i} + \cos\theta\,\uvec{\j}
%  \end{align}
%\end{subequations}
%}
%Resumimos este cambio de base y su inversa en forma matricial
%\begin{subequations}
%\begin{align}
%  \begin{pmatrix}
%    \xhat{r}\\
%    \xhat{\theta}
%  \end{pmatrix}
%  &=
%  \begin{pmatrix}
%    \cos\theta & \sin\theta\\
%    -\sin\theta & \cos\theta
%  \end{pmatrix}
%  \,
%  \begin{pmatrix}
%    \uvec{\i}\\
%    \uvec{\j}
%  \end{pmatrix}\\
%\begin{pmatrix}
%    \uvec{\i}\\
%    \uvec{\j}
%  \end{pmatrix}
%  &=
%  \begin{pmatrix}
%    \cos\theta & -\sin\theta\\
%    \sin\theta & \cos\theta
%  \end{pmatrix}
%  \,
%  \begin{pmatrix}
%    \xhat{r}\\
%    \xhat{\theta}
%  \end{pmatrix} 
%\end{align}
%\end{subequations}
%donde se ha utilizado la inversa de la matriz de cambio de base, que por ser ortogonal,
%la inversa es su transpuesta.
%
%Nótese que los vectores de la base natural $\vvv{e}_r$ y $\vvv{e}_\theta$ se pueden
%escribir en función de los unitarios como
%\begin{subequations}
%\begin{align}
%  \vvv{e}_r &= \xhat{r}\\
%  \vvv{e}_\theta &= r\xhat{\theta}
%\end{align}
%\end{subequations}
%
%\subsubsection{Productos escalares}
%\begin{itemize}
%\item De los vectores unitarios
%{\small
%\begin{subequations}
%  \begin{align}
%    \xhat{r}\cdot\xhat{r}
%    &=
%      (\cos\theta\,\uvec{\i}+\sin\theta\,\uvec{\j})
%      \cdot
%      (\cos\theta\,\uvec{\i}+\sin\theta\,\uvec{\j})
%      = \cos^2\theta + \sin^2\theta = 1\\
%    \xhat{r}\cdot\xhat{\theta}
%    &=
%      \xhat{\theta}\cdot\xhat{r}
%      = (\cos\theta\,\uvec{\i}+\sin\theta\,\uvec{\j})
%      \cdot
%      (-\sin\theta\,\uvec{\i}+\cos\theta\,\uvec{\j})
%      = -\sin\theta\cos\theta + \sin\theta\cos\theta = 0\\
%    \xhat{\theta}\cdot\xhat{\theta}
%    &=
%      (-\sin\theta\,\uvec{\i}+\cos\theta\,\uvec{\j})
%      \cdot
%      (-\sin\theta\,\uvec{\i}+\cos\theta\,\uvec{\j})
%      = \sin^2\theta + \cos^2\theta = 1
%  \end{align}
%\end{subequations}
%}
%
%\item De los vectores $\vvv{e}_r$ y $\vvv{e}_\theta$
%\begin{subequations}
%  \begin{align}\label{eq:R2-erer-productoescalar}
%    \vvv{e}_r\cdot\vvv{e}_r
%    &=
%      \xhat{r}\cdot\vvv{r} = 1\\
%    \label{eq:R2-eretheta-productoescalar}
%    \vvv{e}_r\cdot\vvv{e}_\theta
%    &=
%      \vvv{e}_\theta\cdot\vvv{e}_r
%      = \xhat{r}\cdot r\xhat{\theta} = r \xhat{r}\cdot\xhat{\theta} = 0\\
%    \label{eq:R2-ethetaetheta-productoescalar}
%    \vvv{e}_r\cdot\vvv{e}_\theta
%    &=
%      \vvv{e}_\theta\cdot\vvv{e}_\theta
%      = r\xhat{\theta}\cdot r\xhat{\theta}
%      = r^2 \xhat{\theta}\cdot\xhat{\theta} = r^2\\    
%  \end{align}
%\end{subequations}
%\end{itemize}
%
%\subsubsection{Derivadas de los vectores unitarios}
%Para terminar, nos interesan las derivadas de los vectores $\xhat{r}$ y $\xhat{\theta}$
%con respecto de $r$
%\begin{subequations}
%\begin{align}\label{eq:R2-partial-versor_r-r}
%  \dfrac{\partial\xhat{r}}{\partial r}
%  &=
%  \dfrac{\partial}{\partial r}(\cos\theta\,\uvec{\i}+\sin\theta\,\uvec{\j})
%  = 0\\
%  \label{eq:R2-partial-versor_theta-r}
%  \dfrac{\partial\xhat{\theta}}{\partial r}
%  &=
%  \dfrac{\partial}{\partial r}(-\sin\theta\,\uvec{\i}+\cos\theta\,\uvec{\j})
%  = 0
%\end{align}
%\end{subequations}
%Derivadas con respecto de $\theta$
%\begin{subequations}
%\begin{align}\label{eq:R2-partial-versor_r-theta}
%  \dfrac{\partial\xhat{r}}{\partial\theta}
%  &=
%  \dfrac{\partial}{\partial r}(\cos\theta\,\uvec{\i}+\sin\theta\,\uvec{\j})  
%  = -\sin\theta\,\uvec{\i} + \cos\theta\,\uvec{\j} = \xhat{\theta}\\
%  \label{eq:R2-partial-versor_theta-theta}
%  \dfrac{\partial\xhat{\theta}}{\partial\theta}
%  &=
%  \dfrac{\partial}{\partial r}(-\sin\theta\,\uvec{\i}+\cos\theta\,\uvec{\j})  
%  = -\cos\theta\,\uvec{\i} - \sin\theta\,\uvec{\j} = -\xhat{r}
%\end{align}
%\end{subequations}
%
%\subsection{Métrica y factores de escala}
%Vamos a calcular la métrica y los factores de escala de varias formas:
%\begin{enumerate}
%\item Directamente, utilizando los productos escalares
%  \eqref{eq:R2-erer-productoescalar}, \eqref{eq:R2-eretheta-productoescalar} y
%   \eqref{eq:R2-ethetaetheta-productoescalar}
%  \begin{equation}\label{eq:R2-metrica-polares}
%    \mmm{g}
%    = g_{ij}
%    = \begin{pmatrix}
%      g_{11} & g_{12}\\
%      g_{21} & g_{22}
%      \end{pmatrix}
%    = \begin{pmatrix}
%      \vvv{e}_r\cdot\vvv{e}_r & \vvv{e}_r\cdot\vvv{e}_\theta\\
%      \vvv{e}_\theta\cdot\vvv{e}_r & \vvv{e}_\theta\cdot\vvv{e}_\theta
%    \end{pmatrix}
%    = \begin{pmatrix}
%      1 & 0\\
%      0 & r^2
%    \end{pmatrix}
%  \end{equation}
%  
%\item Empezamos calculando las diferenciales $dx$ y $dy$
%  \begin{subequations}
%    \begin{align}\label{eq:R2-dx}
%      dx
%      &=
%        \dfrac{\partial x}{\partial r} dr + \dfrac{\partial x}{\partial\theta} d\theta
%        = \cos\theta dr - r\sin\theta d\theta\\
%      \label{eq:R2-dy}
%      dy
%      &=
%        \dfrac{\partial y}{\partial r} dr + \dfrac{\partial y}{\partial\theta} d\theta
%        = \sin\theta dr + r\cos\theta d\theta
%    \end{align}
%  \end{subequations}
%
%La métrica se puede deducir a partir del cuadrado de la distancia recorrida en el
%espacio al cambiar las coordenadas
%{\footnotesize
%  \begin{align*}
%    ds^2
%    &=
%      dx^2 + dy^2
%      = (\cos\theta dr - r\sin\theta d\theta)^2
%      + (\sin\theta dr + r\cos\theta d\theta)^2\\
%    &=
%      \cos^2\theta dr^2
%      + \!r^2\sin^2\theta d\theta^2
%      - \!\cancelout{2r\sin\theta \cos\theta dr d\theta}
%      + \!\sin^2\theta dr^2
%      + \!r^2 \cos^2\theta d\theta^2
%      + \!\cancelout{2r\sin\theta \cos\theta dr d\theta}\\
%    &=
%      (\cos^2\theta + \sin^2\theta) dr^2 + r^2(\sin^2\theta + \cos^2\theta) d\theta^2
%      = dr^2 + r^2 d\theta^2
%  \end{align*}
%}
%donde se han tenido en cuenta las ecuaciones~\eqref{eq:R2-dx} y ~\eqref{eq:R2-dy} y que
%los vectores unitarios $\xhat{r}$ y $\xhat{\theta}$ son ortogonales.
%
%\item De forma intuitiva podemos interpretar la métrica. Veamos
%  \begin{itemize}
%  \item Las variables $r$ y $\theta$ son ortogonales, un cambio en una de ellas no afecta
%    a la otra, como se puede comprobar observando las figuras
%    \ref{fig:R2-polares-intuicion-h-r} y \ref{fig:R2-polares-intuicion-h-theta}.
%    De esta manera, vemos que los elementos no diagonales de la métrica valen cero,
%    $g_{12}=g_{21}=0$.
%  \item Por otro lado, en la figura \ref{fig:R2-polares-intuicion-h-r} observamos que
%    cuando modificamos la variable $r$ en una cierta distancia $\Delta r$, estamos
%    desplazándonos en el espacio la misma distancia $\Delta s$. esto significa que
%    el coeficiente de escala $h_r$ vale la unidad. Esto implica que el elemento
%    $g_{11}$ de la métrica vale lo mismo.
%    \begin{subequations}
%      \begin{align}
%        h_r
%        &=
%          \dfrac{\Delta s}{\Delta r} = 1\\[2pt]
%        g_{11}
%        &=
%          h_x^2 = 1
%      \end{align}
%    \end{subequations}
%
%  \item En cuanto al coeficiente de escala $h_\theta$, cuando se modifica $\theta$ en una
%    cierta cantidad $\Delta\theta$, la distancia que se recorre en el espacio es
%    el ángulo por el radio, $\Delta s = r\Delta\theta$. Además, $g_{22} = r^2$. Ver
%    figura \ref{fig:R2-polares-intuicion-h-theta}
%    \begin{subequations}
%      \begin{align}
%        h_\theta
%        &=
%          \dfrac{\Delta s}{\Delta\theta} = \dfrac{r\Delta\theta}{\Delta\theta} = r\\[2pt]
%        g_{22}
%        &=
%          h_\theta^2 = r^2
%      \end{align}
%    \end{subequations}
%    \begin{figure}[ht]
%      \centering
%      \begin{minipage}{0.40\linewidth}
%        % Esta gráfica muestra cómo afecta un cambio en la coordenada r a la distancia
%        % recorrida en el espacio R^2
%        % Escala
%        \def\scl{1.35}
%        % 
%        \pgfmathsetmacro{\EJESEXTRA}{-0.2}
%        \pgfmathsetmacro{\EJESLONG}{2.5}
%        % Eje x
%        \pgfmathsetmacro{\XMLONG}{\EJESEXTRA}
%        \pgfmathsetmacro{\XPLONG}{\EJESLONG}
%        % Eje y
%        \pgfmathsetmacro{\YMLONG}{\EJESEXTRA}
%        \pgfmathsetmacro{\YPLONG}{\EJESLONG}
%        % Punto P original
%        \pgfmathsetmacro{\PMOD}{1.5}
%        \pgfmathsetmacro{\PANG}{30}
%        % Punto P final conservando el mismo ángulo
%        \pgfmathsetmacro{\PFINMOD}{2.4}
%        \pgfmathsetmacro{\PFINANG}{\PANG}      
%        % Fondo
%        \pgfmathsetmacro{\HORZ}{0.25}
%        \pgfmathsetmacro{\VERT}{0.25}
%        % 
%        \centering
%        \begin{tikzpicture}[%
%          scale=\scl,
%          every node/.style={black,font=\small},
%          eje/.style={->},
%          r/.style={line width=.8pt, black},
%          rtxt/.style={black},
%          rfin/.style={line width=1pt, \colorVred},
%          rfintxt/.style={\colorTchg, rotate=\PANG},
%          textooriginal/.style={\colorTorig},
%          pcirculo/.style={black, radius=1pt},
%          angulogirado/.style={fill=\colorAng, draw=\colorGirodos},
%          background/.style={%
%            line width=\bgborderwidth,
%            draw=\bgbordercolor,
%            fill=\bgcolor,
%          },
%          ]
%          % COORDENADAS
%          % Origen
%          \coordinate (O) at (0,0);
%          % Extremo izdo del eje x e inferior del eje y
%          \coordinate (xini) at (\XMLONG cm,0);
%          \coordinate (yini) at (0,\YMLONG cm);
%          % Eje x
%          \path[save path=\ejex] (xini) -- coordinate[pos=0.5] (xtexto)
%          (\XPLONG cm, 0) coordinate (xfin);
%          % Eje y
%          \path[save path=\ejey] (yini) -- coordinate[pos=0.6] (ytexto)
%          (0, \YPLONG cm) coordinate (yfin);
%          % Punto P original
%          \path[save path=\porig] (O) -- coordinate[pos=0.65] (rorigtxt)
%          (\PANG:\PMOD cm) coordinate (P);
%          % Punto P final
%          \path[save path=\pfin] (P) -- coordinate[pos=0.5] (rfintxt)
%          (\PFINANG:\PFINMOD cm) coordinate (P');
%        
%          % DIBUJOS
%          % Ángulo theta
%          \path (xfin) -- (O) -- (P) pic
%          [-{Latex[width=2.5pt,length=3.7pt]},angulogirado,
%          "\scriptsize $\theta$",angle radius=24pt, angle eccentricity=0.72]
%          {angle = xfin--O--P};
%          % Ejes
%          % x
%          \draw[eje, use path=\ejex];
%          \node[right] at (xfin) {$x$};
%          % y
%          \draw[eje, use path=\ejey];
%          \node[above] at (yfin) {$y$};
%          % Coordenada r original
%          \draw[r, use path=\porig];
%          \node[rtxt, above left=-3pt and -1pt] at (rorigtxt) {\footnotesize $r$};
%          % Vector r final
%          \draw[rfin, use path=\pfin];
%          \node[rfintxt, above] at (rfintxt) {\footnotesize $\Delta s = \Delta r$};
%          % Puntos
%          \fill[pcirculo] (P) circle;
%          \fill[pcirculo] (P') circle;
%          \node[rtxt, below right] at (P) {\scriptsize $P$};
%          \node[rtxt, below right] at (P') {\scriptsize $P'$};
%          % Origen
%          \filldraw (O) circle [radius=.4pt];
%          % Fondo amarillo
%          \coordinate (SW)
%          at ($(current bounding box.south west) + (-\HORZ cm,-\VERT cm)$);
%          \coordinate (NE)
%          at ($(current bounding box.north east) + (\HORZ cm,\VERT cm)$);
%          \begin{scope}[on background layer]
%            \draw[background] (SW) rectangle (NE);
%          \end{scope}
%        \end{tikzpicture}
%        \caption{Al cambiar la coordenada $r$, manteniendo constante $\theta$, se recorre
%          una distancia igual en el espacio $\symbb{R}^2$. Por tanto, $h_r=1$ y
%          $g_{11} = 1$.}
%        \label{fig:R2-polares-intuicion-h-r}
%      \end{minipage}
%      \hspace{1em}
%      \begin{minipage}{0.40\linewidth}
%        % Escala
%        \def\scl{1.35}
%        % 
%        \pgfmathsetmacro{\EJESEXTRA}{-0.2}
%        \pgfmathsetmacro{\EJESLONG}{2.5}
%        % Eje x
%        \pgfmathsetmacro{\XMLONG}{\EJESEXTRA}
%        \pgfmathsetmacro{\XPLONG}{\EJESLONG}
%        % Eje y
%        \pgfmathsetmacro{\YMLONG}{\EJESEXTRA}
%        \pgfmathsetmacro{\YPLONG}{\EJESLONG}
%        % Punto P original
%        \pgfmathsetmacro{\PMOD}{2.0}
%        \pgfmathsetmacro{\PANG}{25}
%        % Punto P final conservando el mismo radio
%        \pgfmathsetmacro{\PFINMOD}{\PMOD}
%        \pgfmathsetmacro{\PFINANG}{70}
%        % Texto en arco
%        \pgfmathsetmacro{\PARCANG}{0.5*(\PFINANG) + \PANG)}
%        \pgfmathsetmacro{\PARCMOD}{\PMOD * 1.2}
%        % Fondo
%        \pgfmathsetmacro{\HORZ}{0.25}
%        \pgfmathsetmacro{\VERT}{0.25}
%        % 
%        \centering
%        \begin{tikzpicture}[%
%          scale=\scl,
%          every node/.style={black,font=\small},
%          eje/.style={->},
%          r/.style={line width=.8pt, black},
%          rtxt/.style={black},
%          rfin/.style={line width=1pt, \colorVred},
%          rfintxt/.style={\colorTchg},
%          textooriginal/.style={\colorTorig},
%          pcirculo/.style={black, radius=1pt},
%          angulo/.style={fill=\colorAng, draw=\colorGirodos},
%          angulogirado/.style={fill=\colorAngPink, draw=\colorGiroRed},
%          arco/.style={line width=1pt, draw=\colorGiroRed},
%          background/.style={%
%            line width=\bgborderwidth,
%            draw=\bgbordercolor,
%            fill=\bgcolor,
%          },
%          ]
%
%          % COORDENADAS
%          % Origen
%          \coordinate (O) at (0,0);
%          % Extremo izdo del eje x e inferior del eje y
%          \coordinate (xini) at (\XMLONG cm,0);
%          \coordinate (yini) at (0,\YMLONG cm);
%          % Eje x
%          \path[save path=\ejex] (xini) -- coordinate[pos=0.5] (xtexto)
%          (\XPLONG cm, 0) coordinate (xfin);
%          % Eje y
%          \path[save path=\ejey] (yini) -- coordinate[pos=0.6] (ytexto)
%          (0, \YPLONG cm) coordinate (yfin);
%          % Punto P original
%          \path[save path=\porig] (O) -- coordinate[pos=0.65] (rorigtxt)
%          (\PANG:\PMOD cm) coordinate (P);
%          % Punto P final
%          \path[save path=\pfin] (O) -- coordinate[pos=0.65] (rfintxt)
%          (\PFINANG:\PFINMOD cm) coordinate (P');
%
%          % DIBUJOS
%          % Ángulo theta
%          \path (xfin) -- (O) -- (P) pic
%          [angulo, "\scriptsize $\theta$",angle radius=24pt, angle eccentricity=0.72]
%          {angle = xfin--O--P};
%          % Incremento de theta
%          \path (P) -- (O) -- (P') pic
%          [-{Latex[width=3.0pt,length=4.5pt]}, angulogirado,
%          "\scriptsize $\Delta\theta$",angle radius=24pt, angle eccentricity=0.72]
%          {angle = P--O--P'};
%          % Ejes
%          % x
%          \draw[eje, use path=\ejex];
%          \node[right] at (xfin) {$x$};
%          % y
%          \draw[eje, use path=\ejey];
%          \node[above] at (yfin) {$y$};
%          % Longitud r original
%          \draw[r, use path=\porig];
%          \node[rtxt, below right=-2pt and 0pt] at (rorigtxt) {\footnotesize $r$};
%          % Longitud r final
%          \draw[r, use path=\pfin];
%          \node[rtxt, above left = -2pt and 0pt] at (rfintxt) {\footnotesize $r$};
%          % Arco recorrido
%          \draw[arco] (P) arc (\PANG:\PFINANG:\PMOD)
%          node[rfintxt,above,midway,sloped] {\footnotesize$\Delta s = r\Delta\theta$};
%        
%          % Puntos
%          \fill[pcirculo] (P) circle;
%          \fill[pcirculo] (P') circle;
%          \node[rtxt, below right=-3pt and 0pt] at (P) {\scriptsize $P$};
%          \node[rtxt, above left=0pt and -3pt] at (P') {\scriptsize $P'$};
%          % Origen
%          \filldraw (O) circle [radius=.4pt];
%          % Fondo amarillo
%          \coordinate (SW)
%          at ($(current bounding box.south west) + (-\HORZ cm,-\VERT cm)$);
%          \coordinate (NE)
%          at ($(current bounding box.north east) + (\HORZ cm,\VERT cm)$);
%          \begin{scope}[on background layer]
%            \draw[background] (SW) rectangle (NE);
%          \end{scope}
%
%        \end{tikzpicture}
%        \caption{Al cambiar la coordenada $\theta$, manteniendo $r$ constante, se  recorre
%          una distancia proporcional a $r$ y a $\Delta\theta$. Por tanto, $h_\theta = r$ y
%          $g_{22} = r^2$.}
%        \label{fig:R2-polares-intuicion-h-theta}
%      \end{minipage}
%    \end{figure}
%  \end{itemize}
%  
%\end{enumerate}
%
%\subsection{Gradiente}
%Cálculos preliminares. Se utilizan las expresiones \eqref{eq:R2-x-rtheta},
%\eqref{eq:R2-y-rtheta}, \eqref{eq:R2-r-xy} y \eqref{eq:R2-theta-xy}
%{\small
%\begin{subequations}
%  \begin{align}
%    \dfrac{\partial r}{\partial x}
%    &=
%      \dfrac{1}{\cancelout{2}\sqrt{x^2+y^2}}\cdot \cancelout{2}x
%      = \dfrac{\cancelout{r}\cos\theta}{\cancelout{r}} = \cos\theta\\
%    \dfrac{\partial r}{\partial y}
%    &=
%      \dfrac{1}{\cancelout{2}\sqrt{x^2+y^2}}\cdot \cancelout{2}y
%      = \dfrac{\cancelout{r}\sin\theta}{\cancelout{r}} = \sin\theta\\
%    \dfrac{\partial\theta}{\partial x}
%    &=
%      \dfrac{1}{1+\dfrac{y^2}{x^2}}\,\left(-\dfrac{y}{x}\right)
%      = -\dfrac{x^{\cancelout{\scriptstyle{2}}}}{x^2+y^2}\,\dfrac{y}{\cancelout{x}}
%      = -\dfrac{y}{x^2+y^2}
%      = -\dfrac{\cancelout{r}\sin\theta}{r^{\cancelout{2}}}
%      = -\dfrac{\sin\theta}{r}\\
%    \dfrac{\partial\theta}{\partial x}
%    &=
%      \dfrac{1}{1+\dfrac{y^2}{x^2}}\,\dfrac{1}{x}
%      = \dfrac{x^{\cancelout{\scriptstyle{2}}}}{x^2+y^2}\,\dfrac{1}{\cancelout{x}}
%      = \dfrac{y}{x^2+y^2}
%      = \dfrac{\cancelout{r}\cos\theta}{r^{\cancelout{\scriptstyle{2}}}}
%      = \dfrac{\cos\theta}{r}
%  \end{align}
%\end{subequations}
%}
%{\small
%  \begin{subequations}
%    \begin{align*}
%      \dfrac{\partial f}{\partial x}\,\uvec{\i}
%      &=
%        \left(%
%        \dfrac{\partial f}{\partial r}\dfrac{\partial r}{\partial x}
%        + \dfrac{\partial f}{\partial\theta}\dfrac{\partial\theta}{\partial x}
%        \right)
%        \left(%
%        \cos\theta\,\xhat{r}-\sin\theta\,\xhat{\theta}
%        \right)\\
%      &=
%        \left[%
%        \dfrac{\partial f}{\partial r}\cos\theta
%        + \dfrac{\partial f}{\partial\theta}\left(-\dfrac{\sin\theta}{r}\right)
%        \right]
%        \left(%
%        \cos\theta\,\xhat{r} - \sin\theta\,\xhat{\theta}
%        \right)\\
%      &=
%        \cos^2\theta\dfrac{\partial f}{\partial r}\,\xhat{r}
%        -\dfrac{\sin\theta\cos\theta}{r}\dfrac{\partial f}{\partial\theta}\,\xhat{r}
%        - \sin\theta\cos\theta\dfrac{\partial f}{\partial r}\,\xhat{\theta}
%        + \dfrac{\cos^2\theta}{r}\dfrac{\partial f}{\partial\theta}\,\xhat{\theta}
%    \end{align*}
%  \end{subequations}
%}
%{\small
%  \begin{subequations}
%    \begin{align*}
%      \dfrac{\partial f}{\partial y}\,\uvec{\j}
%      &=
%        \left(%
%        \dfrac{\partial f}{\partial r}\dfrac{\partial f}{\partial y}
%        + \dfrac{\partial f}{\partial\theta}\dfrac{\partial\theta}{\partial y}
%        \right)
%        \left(%
%        \sin\theta\,\xhat{r}-\cos\theta\,\xhat{\theta}
%        \right)\\
%      &=
%        \left[%
%        \dfrac{\partial f}{\partial r}\sin\theta
%        + \dfrac{\partial f}{\partial\theta}\dfrac{\cos\theta}{r}
%        \right]
%        \left(%
%        \sin\theta\,\xhat{r} - \cos\theta\,\xhat{\theta}
%        \right)\\
%      &=
%        \sin^2\theta\dfrac{\partial f}{\partial r}\,\xhat{r}
%        + \dfrac{\sin\theta\cos\theta}{r}\dfrac{\partial f}{\partial\theta}\,\xhat{r}
%        + \sin\theta\cos\theta\dfrac{\partial f}{\partial r}\,\xhat{\theta}
%        + \dfrac{\cos^2\theta}{r}\dfrac{\partial f}{\partial\theta}\,\xhat{\theta}
%    \end{align*}
%  \end{subequations}
%}
%
%El gradiente en polares se obtiene sustituyendo las dos expresiones anteriores en
%\eqref{eq:R2-gradiente-cartesianas} y sumando. A continuación presentamos el resultado
%{\small
%  \begin{equation}\label{eq:R2-gradiente-polares}
%    \vvv{\nabla}\cdot f
%    =
%      \dfrac{\partial f}{\partial x}\,\uvec{\i}
%      + \dfrac{\partial f}{\partial y}\,\uvec{\j}
%      = \dfrac{\partial f}{\partial r}\,\xhat{r}
%      + \dfrac{1}{r}\dfrac{\partial f}{\partial\theta}\,\xhat{\theta}
%  \end{equation}
%}
%
%\subsection{Divergencia}
%La divergencia de un campo vectorial en polares la calcularemos de varias formas pero,
%sin que sirva de precedente, la primera que presentaré no es correcta.
%
%Realizamos un cálculo matricial, similar al realizado en coordenadas cartesianas,
%teniendo en cuenta que en este producto escalar se utiliza la métrica en polares
%\eqref{eq:R2-metrica-polares}
%\begin{equation}
%  \vvv{\nabla}\cdot\vvv{A}
%  \neq
%  \cancelout{%
%    \begin{pmatrix}
%      \partial/\partial r & \partial/\partial\theta      
%    \end{pmatrix}
%    \begin{pmatrix}
%      1 & 0\\
%      0 & r^2
%    \end{pmatrix}
%    \begin{pmatrix}
%      A_r\\
%      A_\theta
%    \end{pmatrix}
%  }
%  = \cancelout{%
%    \dfrac{\partial A_r}{\partial r}
%    + \dfrac{\partial r^2A_\theta}{\partial\theta}
%  }
%\end{equation}
%¿Por qué no es correcto este cálculo, si se ha utilizado la métrica correcta en
%polares, de forma similar a lo que se hizo en cartesianas, ver
%\eqref{eq:R2-divergencia-cartesianas}?
%
%% \hspace{1em}
%No lo es porque, aunque hemos utilizado la métrica, esta no es lo único que cambia
%cuando se pasa de cartesianas a polares, esto es, no solo cambian las distancias
%(la métrica), sino también cambian los vectores de la base con la posición en el plano.
%En este cálculo matricial, y en el que se realizó en cartesianas, estaba implícito el
%que los vectores de la base eran inmutables, lo que es correcto en cartesianas pero no
%lo es en polares, dado que estos cambian con la posición en el plano $\symbb{R}^2$.
%
%En resumen, el sistema de coordenadas polares es curvilíneo ver figura
%\ref{fig:R2-coordenadas-polares}, y esa curvatura hay quetenerla en cuenta.
%
%\begin{figure}[ht]
%  % Escala
%  \def\scl{0.9}
%  % 
%  % Número de círcunferencias
%  \pgfmathsetmacro{\NUMCIRC}{5}
%  % Longitud de un versor
%  \pgfmathsetmacro{\VERSOR}{0.5}
%  % Longitud de línea radial
%  \pgfmathsetmacro{\LINEARADIAL}{\NUMCIRC * \VERSOR + 0.25}
%  % Fondo
%  \pgfmathsetmacro{\HORZ}{0.25}
%  \pgfmathsetmacro{\VERT}{0.25}
%  % 
%  \centering
%  \begin{tikzpicture}[%
%    scale=\scl,
%    every node/.style={black,font=\small},
%    vtexto/.style={font=\footnotesize, pos=1.4},
%    vtheta/.style={-{Latex[round, width=4pt, length=5pt]},
%      line width=0.6pt, \colorVred},
%    vr/.style={vtheta, \colorV},
%    polares/.style={draw=black!25, line width=0.4pt},
%    punto/.style={fill=black, radius=1.25pt},
%    background/.style={%
%      line width=\bgborderwidth,
%      draw=\bgbordercolor,
%      fill=\bgcolor
%    },
%    ]
%    % COORDENADAS
%    % Origen
%    \coordinate (O) at (0,0);
%    
%    % DIBUJOS
%    % Coordenadas polares (Radios de los ángulos)
%    \foreach \angulo / \label
%    in {%
%      0/{0}, 30/{\pi/6}, 60/{\pi/3},
%      90/{\pi/2}, 120/{2\pi/3}, 150/{5\pi/6},
%      180/{\pi}, 210/{7\pi/6}, 240/{4\pi/3},
%      270/{3\pi/2}, 300/{5\pi/3}, 330/{11\pi/6}
%    }
%    {%
%      \draw[polares] (O) -- ++(\angulo:\LINEARADIAL)
%      node[pos=1.12] {\scriptsize $\label$};
%    };
%    \foreach \radio [evaluate=\radio] in {\VERSOR*1,\VERSOR*...,\VERSOR*\NUMCIRC}
%    {%
%      % Circunferencias
%      \draw[polares] (O) circle[radius=\radio cm];
%    };
%    
%    % Vectores unitarios en r=1.0
%    \foreach \radio [evaluate=\radio] in{\VERSOR*1}
%    \foreach \angulo in {30}
%    {%
%      % Vectores unitario radiales
%      \draw[vr] (O) ++(\angulo:\radio) -- ++(\angulo:\VERSOR)
%      node[vtexto, rotate=\angulo-90] {$\xhat{r}$};
%      % Vectores unitarios angulares
%      \draw[vtheta] (O) ++(\angulo:\radio) -- ++(90+\angulo:\VERSOR)
%      node[vtexto, rotate=\angulo-90, \colorTred] {$\xhat{\theta}$};
%      % Posición
%      \fill[punto] (O) ++(\angulo:\radio) circle;
%    };
%    
%    % Vectores unitarios en r=2.0
%    \foreach \radio [evaluate=\radio] in{\VERSOR*2}
%    \foreach \angulo in {150}
%    {%
%      % Vectores unitario radiales
%      \draw[vr] (O) ++(\angulo:\radio) -- ++(\angulo:\VERSOR)
%      node[vtexto, rotate=\angulo-90] {$\xhat{r}$};
%      % Vectores unitarios angulares
%      \draw[vtheta] (O) ++(\angulo:\radio) -- ++(90+\angulo:\VERSOR)
%      node[vtexto, rotate=\angulo-90, \colorTred] {$\xhat{\theta}$};
%      % Posición
%      \fill[punto] (O) ++(\angulo:\radio) circle;
%    };
%    
%    % Vectores unitarios en r=3.0
%    \foreach \radio [evaluate=\radio] in{\VERSOR*3}
%    \foreach \angulo in {210, 330}
%    {%
%      % Vectores unitario radiales
%      \draw[vr] (O) ++(\angulo:\radio) -- ++(\angulo:\VERSOR)
%      node[vtexto, rotate=\angulo-90] {$\xhat{r}$};
%      % Vectores unitarios angulares
%      \draw[vtheta] (O) ++(\angulo:\radio) -- ++(90+\angulo:\VERSOR)
%      node[vtexto, rotate=\angulo-90, \colorTred] {$\xhat{\theta}$};
%      % Posición
%      \fill[punto] (O) ++(\angulo:\radio) circle;
%    };
%    
%    % Vectores unitarios en r=4.0
%    \foreach \radio [evaluate=\radio] in{\VERSOR*4}
%    \foreach \angulo in {30,90,270}
%    {%
%      % Vectores unitario radiales
%      \draw[vr] (O) ++(\angulo:\radio) -- ++(\angulo:\VERSOR)
%      node[vtexto, rotate=\angulo-90] {$\xhat{r}$};
%      % Vectores unitarios angulares
%      \draw[vtheta] (O) ++(\angulo:\radio) -- ++(90+\angulo:\VERSOR)
%      node[vtexto, rotate=\angulo-90, \colorTred] {$\xhat{\theta}$};
%      % Posición
%      \fill[punto] (O) ++(\angulo:\radio) circle;
%    };      
%    
%    % Origen
%    \filldraw (O) circle [radius=.2pt];
%    
%    % Fondo amarillo
%    \coordinate (SW) at ($(current bounding box.south west) + (-\HORZ cm,-\VERT cm)$);
%    \coordinate (NE) at ($(current bounding box.north east) + (\HORZ cm,\VERT cm)$);
%    \begin{scope}[on background layer]
%      \draw[background] (SW) rectangle (NE);
%    \end{scope}
%  \end{tikzpicture}
%  \caption{Coordenadas polares (curvilíneas), donde $r$ representa la distancia al
%    origen (polo) y $\theta$ es el ángulo que forma el vector de posición con el eje
%    $x$, aquí expresado en radianes. Además se muestran algunos vectores unidad
%    $\xhat{r}$ y $\xhat{\theta}$ en distintas posiciones del plano $\symbb{R}^2$.}
%  \label{fig:R2-coordenadas-polares}
%\end{figure}
%¿Entonces, no se puede plantear un enfoque matricial para obtener la divergencia en
%polares? Sí es posible, y lo veremos al final.
%
%\begin{enumerate}
%\item Enfoque basado en la variación de los vectores unitarios.
%  
%  Cuando el operador nabla actúa sobre un campo vectorial como
%  $\vvv{A} = A_r\xhat{r}+A_\theta\xhat{\theta}$, no solo actúa sobre las componentes
%  $A_r$ y $A_\theta$, sino también sobre los vectores unitarios $\xhat{r}$ y
%  $\xhat{\theta}$,
%  ver \eqref{eq:R2-versor-r} y \eqref{eq:R2-versor-theta}. Necesitaremos las derivadas
%  de los versores de la base que se calcularon en \eqref{eq:R2-partial-versor_r-r},
%  \eqref{eq:R2-partial-versor_theta-r}, \eqref{eq:R2-partial-versor_r-theta} y
%  \eqref{eq:R2-partial-versor_theta-theta}, que resumimos a continuación:
%  \begin{itemize}
%  \item Al desplazarnos radialmente, los versores $\xhat{r}$ y $\xhat{\theta}$ no
%    cambian, ver ángulo $\pi/6$\,\unit{\radian} en la figura
%    \ref{fig:R2-coordenadas-polares}
%    \begin{equation}
%      \dfrac{\partial\xhat{r}}{\partial r} = \dfrac{\partial\xhat{\theta}}{\partial r}
%      = 0
%    \end{equation}
%  \item Al rotar infinitesimalmente en sentido antihorario, $\xhat{r}$ cambia según
%    $\xhat{\theta}$, ver por ejemplo, ángulos $\pi/6$\,\unit{\radian} y
%    $\pi/2$\,\unit{\radian} en la figura \ref{fig:R2-coordenadas-polares}
%    \begin{equation}
%      \dfrac{\partial\xhat{r}}{\partial\theta} = \xhat{\theta}
%    \end{equation}
%  \item Al desplazarnos infinitesimalmente en sentido antihorario, $\xhat{\theta}$ cambia
%    según el sentido contrario a $\xhat{r}$, ver por ejemplo, ángulos
%    $\pi/6$\,\unit{\radian} y $\pi/2$\,\unit{\radian} en la
%    figura~\ref{fig:R2-coordenadas-polares}
%    \begin{equation}
%      \dfrac{\partial\xhat{\theta}}{\partial\theta} = -\xhat{r}
%    \end{equation}
%    
%  \item También necesitamos el operador nabla en polares, que podemos obtenerlo a
%    partir del gradiente de un campo escalar en polares \eqref{eq:R2-gradiente-polares}
%    \begin{equation}
%      \vvv{\nabla}
%      = \xhat{r}\,\dfrac{\partial}{\partial r}
%      + \xhat{\theta}\,\dfrac{1}{r}\dfrac{\partial}{\partial\theta}
%    \end{equation}
%  \end{itemize}
%  
%  Con todo lo anterior, planteamos el cálculo de la divergencia de un campo vectorial en
%  polares
%  \[
%    \vvv{\nabla}\cdot\vvv{A}
%    =
%    \left(%
%      \xhat{r}\dfrac{\partial}{\partial r}
%      + \xhat{\theta}\dfrac{\partial}{\partial\theta}
%    \right)
%    \cdot
%    (A_r\xhat{r}+A_\theta\xhat{\theta})
%  \]
%  Obtenemos dos sumandos
%  \begin{equation}\label{eq:R2-divergencia-sumandos}
%    \vvv{\nabla}\cdot\vvv{A}
%    = \xhat{r}\cdot
%    \dfrac{\partial}{\partial r}\left(A_r\xhat{r}+A_\theta\xhat{\theta}\right)
%    + \dfrac{1}{r}\xhat{\theta}\cdot\dfrac{\partial}{\partial\theta}
%    \left(A_r\xhat{r}+A_\theta\xhat{\theta}\right)
%  \end{equation}
%  
%  Desarrollamos el primer sumando, teniendo en cuenta que se anulan, tanto las derivadas
%  parciales con respecto a $r$, como el término que contiene a $\xhat{\theta}$, dado que
%  al multiplicarlo por $\xhat{r}$ dará cero, ya que son ortogonales
%  \begin{align*}
%    \xhat{r}\cdot
%    \dfrac{\partial}{\partial r}\left(A_r\xhat{r}+A_\theta\xhat{\theta}\right)
%    &=
%      \xhat{r}\cdot
%      \left(%
%      \dfrac{\partial A_r}{\partial r}\xhat{r}
%      + \cancelout{A_r\dfrac{\partial\xhat{r}}{\partial r}}
%      + \cancelout{\dfrac{\partial A_\theta}{\partial r}\xhat{\theta}}
%      + \cancelout{A_\theta\dfrac{\partial\xhat{\theta}}{\partial r}}
%      \right)\\
%    &=
%      \dfrac{\partial A_r}{\partial r}\,\xhat{r}\cdot\xhat{r}
%      = \dfrac{\partial A_r}{\partial r}
%  \end{align*}
%  Ahora desarrollamos el último sumando, sustituyendo el valor de las derivadas parciales
%  y teniendo en cuenta que $\xhat{r}$ y $\xhat{\theta}$ son ortogonales
%  \begin{align*}
%    \dfrac{1}{r}\xhat{\theta}\cdot\dfrac{\partial}{\partial\theta}
%    \left(A_r\xhat{r}+A_\theta\xhat{\theta}\right)
%    &=
%      \dfrac{1}{r}\xhat{\theta}\cdot
%      \left(%
%      \cancelout{\dfrac{\partial A_r}{\partial\theta}\xhat{r}}
%      + A_r\dfrac{\partial\xhat{r}}{\partial\theta}
%      + \dfrac{\partial A_\theta}{\partial\theta}\xhat{\theta}
%      + A_\theta\dfrac{\partial\xhat{\theta}}{\partial\theta}
%      \right)\\
%    &= 
%      \dfrac{1}{r}\xhat{\theta}\cdot
%      \left(%
%      A_r\xhat{\theta}
%      + \dfrac{\partial A_\theta}{\partial\theta}\xhat{\theta}
%      + \cancelout{A_\theta\,(-\xhat{r})}
%      \right)\\
%    &=
%      \dfrac{1}{r}
%      \left(%
%      A_r + \dfrac{\partial A_\theta}{\partial\theta}
%      \right)
%      \xhat{\theta}\cdot\xhat{\theta}
%      = \dfrac{1}{r}\dfrac{\partial A_\theta}{\partial\theta}
%      + \dfrac{1}{r}A_r
%  \end{align*}
%  
%  Sustituimos estos resultados en los dos sumandos de la divergencia
%  \eqref{eq:R2-divergencia-sumandos}
%  \[
%    \vvv{\nabla}\cdot\vvv{A}
%    = \dfrac{\partial A_r}{\partial r}
%    + \dfrac{1}{r} A_r
%    + \dfrac{1}{r} \dfrac{\partial A_\theta}{\partial\theta}
%  \]
%  
%  Agrupando términos obtenemos la divergencia en polares
%  \begin{equation}\label{eq:R2-divergencia-polares}
%    \vvv{\nabla}\cdot\vvv{A}
%    = \dfrac{1}{r}\dfrac{\partial}{\partial r}(rA_r)
%    + \dfrac{1}{r}\dfrac{\partial A_\theta}{\partial\theta}
%  \end{equation}
%
%\item Cálculo utilizando la métrica y válido en polares y en espacios curvos
%  \begin{equation}
%    \vvv{\nabla}\cdot\vvv{A}
%    = \dfrac{1}{\sqrt{g}}\,\partial_i\big(\sqrt{g} A^i\big)
%  \end{equation}
%  Aplicando la fórmula, sabiendo que la raíz cuadrada del valor absoluto de la métrica
%  es $g=r$ y que $r$ es independiente de $\theta$
%  \begin{align*}
%    \vvv{\nabla}\cdot\vvv{A}
%    &=
%      \dfrac{1}{r}\,\left[%
%      \partial_r\!\left(rA^r\right) + \partial_\theta\!\left(rA^\theta\right)
%      \right]
%    =
%      \dfrac{1}{r}\,\left[%
%      \partial_r\!\left(rA^r\right) + r\,\partial_\theta A^\theta
%      \right]      
%  \end{align*}
%  donde $A^r$ y $A^\theta$ son las componentes contravariantes del vector. En física
%  normalmente se usan las componentes covariantes $A_r$ y $A_\theta$.
%
%  Calculamos la inversa del tensor métrico, que como es diagonal, la inversa de cada
%  elemento es el recíproco de los elementos diagonales
%  \begin{equation}
%    \mmm{g} = \begin{pmatrix}1 & 0\\ 0 & 1/r^2 \end{pmatrix}
%  \end{equation}
%  Para bajar el índice, multiplicamos por la inversa del  tensor métrico $A^i = g^{ij}
%  A_j$.
%  Así, desarrollando el sumatorio
%  \begin{align*}
%    A^r
%    &=
%      g^{rr}A_r + g^{r\theta}A_\theta = 1\cdot A_r + 0\cdot A_\theta = A_r\\
%    A^\theta
%    &=
%      g^{\theta r}A_r + g^{\theta\theta}A_\theta = 0\cdot A_r + \dfrac{1}{r^2} A_\theta
%      = \dfrac{1}{r^2}A_\theta
%  \end{align*}
%  Sustituimos en
%  \begin{align*}
%    \vvv{\nabla}\cdot\vvv{A}
%    = \dfrac{1}{r}\left[%
%    \dfrac{\partial}{\partial r}\!\left(r\,A_r\right)
%    + r\, \partial_\theta\left(\dfrac{1}{r^2}\, A_\theta\right)
%    \right]
%    = \dfrac{1}{r}\left[%
%    \dfrac{\partial}{\partial r}\!\left(r\,A_r\right)
%    + \dfrac{1}{r}\, \partial_\theta A_\theta
%    \right]
%  \end{align*}
%  
%\end{enumerate}
%
%
%\subsection{Laplaciana} 
%
%\subsection{Elemento de área en coordenadas polares planas}
%\begin{align*}
%  d^2a
%  &=
%    \dfrac{\partial(x,y)}{\partial(r,\theta)} dr\,d\theta
%  = \begin{vmatrix}
%    \partial x/\partial r & \partial x/\partial\theta\\
%    \partial y/\partial r & \partial y/\partial\theta
%  \end{vmatrix}
%  dr\,d\theta
%  = \begin{vmatrix}
%    \cos\theta & -r\sin\theta\\
%    \sin\theta & r\cos\theta
%  \end{vmatrix}
%    dr\,d\theta\\
%  &=
%    (r\cos^2\theta + r\sin^2\theta)\,dr\,d\theta
%    = r (\cos^2\theta + \sin^2\theta)\,dr\,d\theta
%\end{align*}
%donde hemos utilizado las relaciones \eqref{eq:R2-x-rtheta} y \eqref{eq:R2-y-rtheta}.
%Simplificando finalmente, queda
%\begin{equation}
%  d^2a = rdr\,d\theta
%\end{equation}



%%% Local Variables:
%%% mode: latex
%%% TeX-engine: luatex
%%% TeX-master: "../retazosmatematicas.tex"
%%% End:
