% coord_polares_R2.tex
%
% Copyright (C) 2022--2026 José A. Navarro Ramón <janr.devel@gmail.com>
% Licencia del código GPLv2
% Licencia Creative Commons Recognition Non-Commercial Share-alike.
% (CC-BY-NC-SA)

\chapter{Coordenadas polares planas (espacio euclídeo
  \mathinhead{\symbb{R}^2}{espr2})}
\section{Coordenadas cartesianas}

Vector de posición de cualquier punto del espacio $\symbb{R}^2$ en coordenadas
cartesianas
\begin{equation}\label{eq:R2-r-cartesianas}
  \vvv{r}
  = x\,\uvec{\i} + y\,\uvec{\j}
\end{equation}
\begin{figure}[ht]
  % Escala
  \def\scl{1}
  %
  \pgfmathsetmacro{\EJESEXTRA}{-0.2}
  \pgfmathsetmacro{\EJESLONG}{2.3}
  % Eje x
  \pgfmathsetmacro{\XMLONG}{\EJESEXTRA}
  \pgfmathsetmacro{\XPLONG}{\EJESLONG}
  % Eje y
  \pgfmathsetmacro{\YMLONG}{\EJESEXTRA}
  \pgfmathsetmacro{\YPLONG}{\EJESLONG}
  % Versores
  \pgfmathsetmacro{\VMOD}{0.7}
  % Vector P
  \pgfmathsetmacro{\PMOD}{2.5}
  \pgfmathsetmacro{\PANG}{40}
  % Fondo
  \pgfmathsetmacro{\HORZ}{0.25}
  \pgfmathsetmacro{\VERT}{0.25}
  % 
  \centering
  \begin{tikzpicture}[%
    scale=\scl,
    every node/.style={black,font=\small},
    eje/.style={->},
    proyeccion/.style={black!60, densely dotted},
    vector/.style={-{Latex[round]}, shorten >=1.2pt, line width=.8pt,\colorV},
    versor/.style={-{Latex[round, width=4pt, length=8pt]}, line width=1pt, \colorVred},
    textooriginal/.style={\colorTorig},
    pcirculo/.style={fill=\colorPorig, draw=black},
    angulogirado/.style={fill=\colorAngGreen, draw=\colorGirodos},
    background/.style={%
      line width=\bgborderwidth,
      draw=\bgbordercolor,
      fill=\bgcolor,
    },
    ]
    % COORDENADAS
    % Origen
    \coordinate (O) at (0,0);
    % Extremo izdo del eje x e inferior del eje y
    \coordinate (xini) at (\XMLONG cm,0);
    \coordinate (yini) at (0,\YMLONG cm);
    \coordinate (ifin) at (\VMOD cm, 0);
    \coordinate (jfin) at (0, \VMOD cm);
    % Eje x
    \path[save path=\ejex] (xini) -- coordinate[pos=0.55] (xtexto) (\XPLONG cm, 0)
    coordinate (xfin);
    % Eje y
    \path[save path=\ejey] (yini) -- coordinate[pos=0.55] (ytexto) (0, \YPLONG cm)
    coordinate (yfin);
    % Versores i y j
    \path[save path=\i] (O) -- coordinate[pos=0.5] (itexto) (ifin);
    \path[save path=\j] (O) -- coordinate[pos=0.5] (jtexto) (jfin);    
    % Vector
    \path[save path=\vector] (O) -- coordinate[pos=0.5] (pmid) (\PANG:\PMOD cm)
    coordinate (P);
    % Proyección de P en los ejes
    \path[save path=\proyx] (P) |- (xfin);
    \path[save path=\proyy] (P) -| (yfin);

    % DIBUJOS
    % Proyección de P en los ejes
    \draw[proyeccion, use path=\proyx];
    \draw[proyeccion, use path=\proyy];
    % Ejes
    % x
    \draw[eje, use path=\ejex];
    \node[right] at (xfin) {$x$};
    \node[below] at (xtexto) {$x$};
    % y
    \draw[eje, use path=\ejey];
    \node[above] at (yfin) {$y$};
    \node[left] at (ytexto) {$y$};
    % Versores
    \draw[versor, use path=\i];
    \node[below, \colorTred] at (itexto) {\footnotesize $\uvec{\i}$};
    \draw[versor, use path=\j];
    \node[left, \colorTred] at (jtexto) {\footnotesize $\uvec{\j}$};
    % Vector y punto  P
    \draw[vector, use path=\vector];
    \node[above left=-3pt and -1pt] at (pmid) {$\vvv{r}$};
    \filldraw[fill=green, draw=black] (P) circle[radius=1.2pt];
    \node[above] at (P) {$P$};
    
    % Origen
    \filldraw (O) circle [radius=.2pt];
    % Fondo amarillo
    \coordinate (SW) at ($(current bounding box.south west) + (-\HORZ cm,-\VERT cm)$);
    \coordinate (NE) at ($(current bounding box.north east) + (\HORZ cm,\VERT cm)$);    
    \begin{scope}[on background layer]
      \draw[background] (SW) rectangle (NE);
    \end{scope}
  \end{tikzpicture}
  \caption{Coordenadas cartesianas de un punto $P$ del plano y vectores unitarios.}
  \label{fig:R2-coordenadas-cartesianas-R2}
\end{figure}

\subsection{Base, vectores unitarios y su producto escalar}
\subsubsection{Base y vectores unitarios}
Si derivamos el vector de posición anterior \eqref{eq:R2-r-cartesianas}, con respecto a
cada una de las coordenadas, obtenemos los vectores de una  posible base en estas
coordenadas. Estos vectores llevan información concreta acerca de cómo varia la posición
en el espacio cuando variamos cada una de las coordenadas ---tal y como sugieren las
derivadas---, lo que será importante para obtener la métrica en estas coordenadas.
\begin{subequations}
  \begin{align}\label{eq:R2-ex}
    \vvv{e}_x
    &=
      \dfrac{\partial\vvv{r}}{\partial x}
      = \uvec{\i}\\
    \label{eq:R2-ey}
    \vvv{e}_y
    &= \dfrac{\partial\vvv{r}}{\partial y}
      = \uvec{\j}
  \end{align}
\end{subequations}
Estos vectores de la base son importantes para calcular la métrica en cartesianas
\begin{equation}\label{eq:R2-basefundamental}
  B=\set{\vvv{e}_x, \vvv{e}_y} = \set{\uvec{\i},\, \uvec{\j}}
\end{equation}
Hacemos notar que los vectores de esta base son unitarios tradicionales en este sistema
de coordenadas, $\uvec{\i}$ y $\uvec{\j}$.

\subsubsection{Producto escalar de los vectores de la base}
Los vectores de la base fundamental son vectores unitarios.
La base normalizada en coordenadas cartesianas en este espacio está formada por los vectores $\uvec{\i}$ y $\uvec{\j}$, en los ejes $x$ e $y$, respectivamente.
A continuación resumimos el valor de los productos escalares entre ellos
\begin{subequations}
  \begin{align}
    \label{eq:R2-ii-producto-escalar}
    \vvv{e}_x\cdot\vvv{e}_x = \uvec{\i}\cdot\uvec{\i}
    &= |\uvec{\i}|\,|\uvec{\i}|\cos 0 = 1\\
    \label{eq:R2-ij-producto-escalar}    
    \vvv{e}_x\cdot\vvv{e}_y = \uvec{\i}\cdot\uvec{\j}
    &= |\uvec{\i}|\,|\uvec{\j}|\cos\pi/2 = 0\\
    \label{eq:R2-ji-producto-escalar} 
    \vvv{e}_y\cdot\vvv{e}_x = \uvec{\j}\cdot\uvec{\i}
    &= |\uvec{\j}|\,|\uvec{\i}|\cos\pi/2 = 0\\
    \label{eq:R2-jj-producto-escalar}
    \vvv{e}_y\cdot\vvv{e}_y = \uvec{\j}\cdot\uvec{\j}
    &= |\uvec{\j}|\,|\uvec{\j}|\cos 0 = 1
  \end{align}
\end{subequations}

\subsection{Métrica y factores de escala}
La métrica en cartesianas se calcula utilizando los vectores de la base calculada en
\eqref{eq:R2-ex} y \eqref{eq:R2-ey}. Los productos escalares se calcularon en
\eqref{eq:R2-ii-producto-escalar}, \eqref{eq:R2-ij-producto-escalar},
\eqref{eq:R2-ji-producto-escalar} y \eqref{eq:R2-jj-producto-escalar}
\begin{equation}\label{eq:R2-metrica-cartesianas}
  \mmm{g}
  = g_{ij}
  = \begin{pmatrix}
    g_{ii} & g_{ij}\\
    g_{ji} & g_{jj}
    \end{pmatrix}
  = \begin{pmatrix}
      \vvv{e}_x\cdot\vvv{e}_x & \vvv{e}_x\cdot\vvv{e}_y\\
      \vvv{e}_y\cdot\vvv{e}_x & \vvv{e}_y\cdot\vvv{e}_y
  \end{pmatrix}
  = \begin{pmatrix}
      \uvec{\i}\cdot\uvec{\i} & \uvec{\i}\cdot\uvec{\j}\\
      \uvec{i}\cdot\uvec{j} & \uvec{\j}\cdot\uvec{\j}
  \end{pmatrix}
  = \begin{pmatrix}
    1 & 0\\
    0 & 1
  \end{pmatrix}
\end{equation}

Se llaman \emph{factores de escala} o \emph{coeficientes métricos} a la raíz cuadrada de
los elementos diagonales en la métrica en un sistema de coordenadas ortogonales, como el
que nos ocupa. En coordenadas cartesianas estándar serían
\begin{align*}
  h_x &= \sqrt{g_{11}} = \sqrt{1} = 1\\
  h_y &= \sqrt{g_{22}} = \sqrt{1} = 1  
\end{align*}

Fijémonos en la métrica \eqref{eq:R2-metrica-cartesianas}
\begin{itemize}
\item Los elementos no diagonales son nulos. Esto se interpreta como que los vectores
  unitarios $\uvec{\i}$ y $\uvec{\j}$ son independientes, esto es, son ortogonales.
  Así, un cambio en una coordenada no afecta a la otra. Así, si nos desplazamos una
  distancia cualquiera en el eje $x$, no habremos recorrido ninguna distancia en el eje
  $y$, y viceversa, por lo que $h_{xy}=h_{yx}=0$.
\item Los elementos diagonales valen la unidad. Cada elemento diagonal de la métrica que
  vale la unidad implica que un cambio en la coordenada respectiva implica una distancia
  recorrida de una unidad en el espacio. En realidad, el espacio recorrido es el factor
  de escala correspondiente, pero en este caso da lo mismo
  \begin{subequations}
    \begin{align}
      \Delta s &= h_x \Delta x = \sqrt{1}\, \Delta x = \Delta x\\
      \Delta s &= h_y \Delta y = \sqrt{1}\, \Delta y = \Delta y
    \end{align}
  \end{subequations}
  Para adquirir algo de intuición, imagina que cambiamos la posición en el eje $x$
  cualquier distancia, digamos un metro, es fácil entender que en el espacio nos habremos
  desplazado la misma distancia. Por tanto es lógico razonar que el factor de escala
  $h_x$ vale 1. Lo mismo podemos decir para el factor de escala en el eje $y$, $h_y=1$.
\end{itemize}

\subsection{Gradiente, divergencia y laplaciana}
Presentamos un resumen de estas operaciones
\begin{itemize}
\item El gradiente de un campo escalar $\phi(x,y)$ es un vector
\begin{equation}
  \vvv{\nabla}\phi
  = \dfrac{\partial \phi}{\partial x}\,\uvec{\i}
  + \dfrac{\partial \phi}{\partial y}\,\uvec{\j}
  = \begin{pmatrix}
    \partial \phi/\partial x\\
    \partial \phi/\partial y
    \end{pmatrix}
\end{equation}
El vector gradiente de $\phi$ evaluado en un punto genérico $(x,y)$ del dominio de $\phi$
indica la dirección en la cual el campo $\phi$ varía más rápidamente y su módulo
representa el ritmo de variación de $\phi$ en la dirección de dicho vector gradiente.

El operador gradiente se comporta como un vector bajo rotaciones
\begin{equation}
  \vvv{\nabla}
  = \dfrac{\partial}{\partial x}\,\uvec{\i}
  + \dfrac{\partial}{\partial y}\,\uvec{\j}
  = \begin{pmatrix}
    \partial/\partial x\\
    \partial/\partial y
    \end{pmatrix}
\end{equation}

\item La divergencia de un campo vectorial $\vvv{A}=(A_x, A_y)$ es un escalar
  \begin{equation}\label{eq:R2-divergencia-cartesianas}
    \vvv{\nabla}\cdot\vvv{A}
    = \begin{pmatrix}
      \partial/\partial x & \partial/\partial y      
    \end{pmatrix}
    \begin{pmatrix}
      1 & 0\\
      0 & 1
    \end{pmatrix}
    \begin{pmatrix}
      A_x\\
      A_y
    \end{pmatrix}
    = \dfrac{\partial A_x}{\partial x}
    + \dfrac{\partial A_y}{\partial y}
  \end{equation}
  La divergencia mide la diferencia entre el flujo saliente y el flujo entrante de un
  campo vectorial sobre la superficie que rodea a un volumen de control, por tanto, si el
  campo tiene ''fuentes'' la divergencia será positiva, y si tiene "sumideros", la
  divergencia será negativa. La divergencia mide la rapidez neta con la que se conduce la
  materia al exterior de cada punto, y en el caso de ser la divergencia idénticamente
  igual a cero, describe al flujo incompresible del fluido, llamado también campo
  solenoidal.  
\item La laplaciana de un campo escalar $f(x,y)$ es la divergencia del gradiente y es un
  escalar
  \begin{equation}\label{eq:R2-gradiente-cartesianas}
    \Delta \phi
    =
    \nabla^2 \phi
    = \vvv{\nabla}\cdot\vvv{\nabla}\phi
    = \dfrac{\partial^2 \phi}{\partial x^2}
    + \dfrac{\partial^2 \phi}{\partial y^2}
  \end{equation}
Mide la concentración de un campo en un punto. Si es negativo, indica una tendencia a la
concentración (fuente); si es positivo, indica dispersión.

El operador laplaciano es
\begin{equation}
  \nabla^2
  = \dfrac{\partial^2}{\partial x^2}
  + \dfrac{\partial^2}{\partial y^2}
\end{equation}
\end{itemize}

\subsection{Elemento de área}
El elemento diferencial de área en cartesianas es
\begin{equation}
  d^2a = dx\,dy
\end{equation}

Este elemento se relaciona trivialmente con el jacobiano de la
\emph{transformación identidad} $f(x,y) \longrightarrow f(x,y)$
\[
  d^2a = \dfrac{\partial(x,y)}{\partial(x,y)} dx dy
  = \begin{vmatrix}
    \partial x/\partial x & \partial x/\partial y\\
    \partial y/\partial x & \partial y/\partial y
  \end{vmatrix}
  dx\,dy
  = \begin{vmatrix}
    1 & 0\\
    0 & 1
  \end{vmatrix}
  dx\,dy
  = 1\,dx\,dy
  = dx\,dy
\]


\section{Coordenadas polares planas}
En este sistema, se utilizan dos coordenadas, $r$ y $\theta$. La primera coordenada es la
distancia entre el origen de coordenadas y un punto del espacio, y la segunda es el
ángulo que forma el eje de abcisas con la línea que une el origen de coordenadas con el
punto, ver figura \ref{fig:R2-coordenadas-R2}.

\begin{figure}[ht]
  \centering
  \begin{minipage}{0.40\linewidth}
    % Escala
    \def\scl{1.25}
    % 
    \pgfmathsetmacro{\EJESEXTRA}{-0.2}
    \pgfmathsetmacro{\EJESLONG}{2.3}
    % Eje x
    \pgfmathsetmacro{\XMLONG}{\EJESEXTRA}
    \pgfmathsetmacro{\XPLONG}{\EJESLONG}
    % Eje y
    \pgfmathsetmacro{\YMLONG}{\EJESEXTRA}
    \pgfmathsetmacro{\YPLONG}{\EJESLONG}
    % Vector P
    \pgfmathsetmacro{\PMOD}{2.5}
    \pgfmathsetmacro{\PANG}{40}
    % Fondo
    \pgfmathsetmacro{\HORZ}{0.25}
    \pgfmathsetmacro{\VERT}{0.25}
    % 
    \centering
    \begin{tikzpicture}[%
      scale=\scl,
      every node/.style={black,font=\small},
      eje/.style={->},
      proyeccion/.style={black!60, densely dotted},
      longitud/.style={line width=.5pt,\colorV},
      textooriginal/.style={\colorTorig},
      pcirculo/.style={fill=\colorPorig, draw=black},
      angulogirado/.style={fill=\colorAngGreen, draw=\colorGirodos},
      background/.style={%
        line width=\bgborderwidth,
        draw=\bgbordercolor,
        fill=\bgcolor,
      },
      ]
      % COORDENADAS
      % Origen
      \coordinate (O) at (0,0);
      % Extremo izdo del eje x e inferior del eje y
      \coordinate (xini) at (\XMLONG cm,0);
      \coordinate (yini) at (0,\YMLONG cm);
      % Eje x
      \path[save path=\ejex] (xini) -- coordinate[pos=0.5] (xtexto) (\XPLONG cm, 0)
      coordinate (xfin);
      % Eje y
      \path[save path=\ejey] (yini) -- coordinate[pos=0.6] (ytexto) (0, \YPLONG cm)
      coordinate (yfin);
      % Vector
      \path[save path=\r] (O) -- coordinate[pos=0.5] (pmid) (\PANG:\PMOD cm)
      coordinate (P);
      % Proyección de P en los ejes
      \path[save path=\proyx] (P) |- (xfin);
      \path[save path=\proyy] (P) -| (yfin);

      % DIBUJOS
      % Ángulo theta
      \path (xfin) -- (O) -- (P) pic
      [-{Latex[width=2.5pt,length=3.7pt]},angulogirado,
      "\footnotesize $\theta$",angle radius=20pt, angle eccentricity=0.68]
      {angle = xfin--O--P};
      % Proyección de P en los ejes
      \draw[proyeccion, use path=\proyx];
      \draw[proyeccion, use path=\proyy];
      % Ejes
      % x
      \draw[eje, use path=\ejex];
      \node[right] at (xfin) {$x$};
      %\node[below] at (xtexto) {$r\cos\theta$};
      % y
      \draw[eje, use path=\ejey];
      \node[above] at (yfin) {$y$};
      %\node[rotate=90, left=8pt] at (ytexto) {$r\sin\theta$};
      % Coordenada r y punto  P
      \draw[longitud, use path=\r];
      \node[above left=-3pt and -1pt] at (pmid) {$r$};
      \filldraw[fill=green, draw=black] (P) circle[radius=1.2pt];
      \node[above] at (P) {$P$};

      % Origen
      \filldraw (O) circle [radius=.2pt];
      % Fondo amarillo
      \coordinate (SW) at ($(current bounding box.south west) + (-\HORZ cm,-\VERT cm)$);
      \coordinate (NE) at ($(current bounding box.north east) + (\HORZ cm,\VERT cm)$);
      \begin{scope}[on background layer]
        \draw[background] (SW) rectangle (NE);
      \end{scope}
    \end{tikzpicture}
  \caption{Coordenadas polares $r$ y $\theta$ de un punto $P$ del plano.}
  \label{fig:R2-coordenadas-R2}
\end{minipage}
\hspace{1em}
\begin{minipage}{0.40\linewidth}
    % Escala
    \def\scl{1.20}
    % 
    \pgfmathsetmacro{\EJESEXTRA}{-0.2}
    \pgfmathsetmacro{\EJESLONG}{2.3}
    % Eje x
    \pgfmathsetmacro{\XMLONG}{\EJESEXTRA}
    \pgfmathsetmacro{\XPLONG}{\EJESLONG}
    % Eje y
    \pgfmathsetmacro{\YMLONG}{\EJESEXTRA}
    \pgfmathsetmacro{\YPLONG}{\EJESLONG}
    % Vector P
    \pgfmathsetmacro{\PMOD}{2.5}
    \pgfmathsetmacro{\PANG}{40}
    % Fondo
    \pgfmathsetmacro{\HORZ}{0.25}
    \pgfmathsetmacro{\VERT}{0.25}
    % 
    \centering
    \begin{tikzpicture}[%
      scale=\scl,
      every node/.style={black,font=\small},
      eje/.style={->},
      proyeccion/.style={black!60, densely dotted},
      vector/.style={-{Latex[round]}, shorten >=1.2pt, line width=.8pt,\colorV},
      textooriginal/.style={\colorTorig},
      pcirculo/.style={fill=\colorPorig, draw=black},
      angulogirado/.style={fill=\colorAngGreen, draw=\colorGirodos},
      background/.style={%
        line width=\bgborderwidth,
        draw=\bgbordercolor,
        fill=\bgcolor,
      },
      ]
      % COORDENADAS
      % Origen
      \coordinate (O) at (0,0);
      % Extremo izdo del eje x e inferior del eje y
      \coordinate (xini) at (\XMLONG cm,0);
      \coordinate (yini) at (0,\YMLONG cm);
      % Eje x
      \path[save path=\ejex] (xini) -- coordinate[pos=0.5] (xtexto) (\XPLONG cm, 0)
      coordinate (xfin);
      % Eje y
      \path[save path=\ejey] (yini) -- coordinate[pos=0.6] (ytexto) (0, \YPLONG cm)
      coordinate (yfin);
      % Vector
      \path[save path=\vector] (O) -- coordinate[pos=0.5] (pmid) (\PANG:\PMOD cm)
      coordinate (P);
      % Proyección de P en los ejes
      \path[save path=\proyx] (P) |- (xfin);
      \path[save path=\proyy] (P) -| (yfin);

      % DIBUJOS
      % Ángulo theta
      \path (xfin) -- (O) -- (P) pic
      [-{Latex[width=2.5pt,length=3.7pt]},angulogirado,
      "\footnotesize $\theta$",angle radius=20pt, angle eccentricity=0.68]
      {angle = xfin--O--P};
      % Proyección de P en los ejes
      \draw[proyeccion, use path=\proyx];
      \draw[proyeccion, use path=\proyy];
      % Ejes
      % x
      \draw[eje, use path=\ejex];
      \node[right] at (xfin) {$x$};
      \node[below] at (xtexto) {$r\cos\theta$};
      % y
      \draw[eje, use path=\ejey];
      \node[above] at (yfin) {$y$};
      \node[rotate=90, left=8pt] at (ytexto) {$r\sin\theta$};
      % Vector y punto  P
      \draw[vector, use path=\vector];
      \node[above left=-3pt and -1pt] at (pmid) {$\vvv{r}$};
      \filldraw[fill=green, draw=black] (P) circle[radius=1.2pt];
      \node[above] at (P) {$P$};

      % Origen
      \filldraw (O) circle [radius=.2pt];
      % Fondo amarillo
      \coordinate (SW) at ($(current bounding box.south west) + (-\HORZ cm,-\VERT cm)$);
      \coordinate (NE) at ($(current bounding box.north east) + (\HORZ cm,\VERT cm)$);
      \begin{scope}[on background layer]
        \draw[background] (SW) rectangle (NE);
      \end{scope}
    \end{tikzpicture}
    \caption{Vector de posición $\vvv{r}$, y coordenadas $x$ e $y$ en función de $r$ y
      $\theta$.}
  \label{fig:R2:vector-r-R2}
\end{minipage}
\end{figure}

\subsection{Relación con las coordenadas cartesianas}
De la figura \ref{fig:R2:vector-r-R2} se desprenden las siguientes relaciones
\begin{subequations}
  \begin{align}\label{eq:R2-x-rtheta}
    x &= r\cos\theta\\
    \label{eq:R2-y-rtheta}
    y &= r\sin\theta
  \end{align}
\end{subequations}
y, sus inversas
\begin{subequations}
  \begin{align}\label{eq:R2-r-xy}
    r &= \sqrt{x^2 + y^2}\\
    \label{eq:R2-theta-xy}
    \theta &= \arctan\left(\frac{y}{x}\right)
  \end{align}
\end{subequations}

\subsection{Bases, vectores unitarios, sus productos escalares y derivadas}
El vector $\vvv{r}$ que representa cualquier punto de $\symbb{R}^2$ se puede escribir, en
la base estándar de coordenadas cartesianas, como
\begin{equation}
  \vvv{r} = r\cos\theta\,\uvec{\i} + r\sin\theta\,\uvec{\j}
\end{equation}
\subsubsection{Derivadas parciales de la posición}
Los vectores ---no necesariamente unitarios--- que llevan información del cambio de
posición al variar cada coordenada forman una base $\set{\vvv{e}_r, \vvv{e}_\theta}$ y se
calculan derivando parcialmente el vector $\vvv{r}$ con respecto de las variables $r$ y
$\theta$
{\small
\begin{subequations}
\begin{align}
  \vvv{e}_r
  &=
    \dfrac{\partial\vvv{r}}{\partial r}
    = \cos\theta\,\uvec{\i} + \sin\theta\,\uvec{\j}\\
  \vvv{e}_\theta
  &=
    \dfrac{\partial\vvv{r}}{\partial\theta}
    = -r\sin\theta\,\uvec{\i} + r\cos\theta\,\uvec{\j}
\end{align}
\end{subequations}
}








































Estos vectores son los que se utilizarán para calcular la métrica en coordenadas polares.

\subsubsection{Vectores unitarios}
Calculamos los vectores unitarios en coordenadas polares, $\xhat{r}$ y $\xhat{\theta}$
{\small
\begin{subequations}
  \begin{align}
    \xhat{r}
    &= \dfrac{\vvv{e}_r}{|\vvv{e}_r|}
    =\dfrac{\cos\theta\,\uvec{\i}+\sin\theta\,\uvec{\j}}
    {\sqrt{\cos^2\theta+\sin^2\theta}}
    = \dfrac{\cos\theta\,\uvec{\i}+\sin\theta\,\uvec{\j}}{1}
  = \cos\theta\,\uvec{\i} + \sin\theta\,\uvec{\j}\\
  \xhat{\theta}
  &= \dfrac{\vvv{e}_\theta}{|\vvv{e}_\theta|}
  = \dfrac{-r\sin\theta\,\uvec{\i}+\cos\theta\,\uvec{\j}}
  {\sqrt{r^2(\cos^2\theta+\sin^2\theta)}}
  = \dfrac{-r\sin\theta\,\uvec{\i}+r\cos\theta\,\uvec{\j}}{r}
  = -\sin\theta\,\uvec{\i} + \cos\theta\,\uvec{\j}
  \end{align}
\end{subequations}
}
Resumimos este cambio de base y su inversa en forma matricial
\begin{subequations}
\begin{align}
  \begin{pmatrix}
    \xhat{r}\\
    \xhat{\theta}
  \end{pmatrix}
  &=
  \begin{pmatrix}
    \cos\theta & \sin\theta\\
    -\sin\theta & \cos\theta
  \end{pmatrix}
  \,
  \begin{pmatrix}
    \uvec{\i}\\
    \uvec{\j}
  \end{pmatrix}\\
\begin{pmatrix}
    \uvec{\i}\\
    \uvec{\j}
  \end{pmatrix}
  &=
  \begin{pmatrix}
    \cos\theta & -\sin\theta\\
    \sin\theta & \cos\theta
  \end{pmatrix}
  \,
  \begin{pmatrix}
    \xhat{r}\\
    \xhat{\theta}
  \end{pmatrix} 
\end{align}
\end{subequations}
donde se ha utilizado la inversa de la matriz de cambio de base, que por ser ortogonal,
la inversa es su transpuesta.

Nótese que los vectores $\vvv{e}_r$ y $\vvv{e}_\theta$ se pueden escribir como
\begin{subequations}
\begin{align}
  \vvv{e}_r &= \xhat{r}\\
  \vvv{e}_\theta &= r\xhat{\theta}
\end{align}
\end{subequations}

\subsubsection{Productos escalares}
\begin{itemize}
\item De los vectores unitarios
{\small
\begin{subequations}
  \begin{align}
    \xhat{r}\cdot\xhat{r}
    &=
      (\cos\theta\,\uvec{\i}+\sin\theta\,\uvec{\j})
      \cdot
      (\cos\theta\,\uvec{\i}+\sin\theta\,\uvec{\j})
      = \cos^2\theta + \sin^2\theta = 1\\
    \xhat{r}\cdot\xhat{\theta}
    &=
      \xhat{\theta}\cdot\xhat{r}
      = (\cos\theta\,\uvec{\i}+\sin\theta\,\uvec{\j})
      \cdot
      (-\sin\theta\,\uvec{\i}+\cos\theta\,\uvec{\j})
      = -\sin\theta\cos\theta + \sin\theta\cos\theta = 0\\
    \xhat{\theta}\cdot\xhat{\theta}
    &=
      (-\sin\theta\,\uvec{\i}+\cos\theta\,\uvec{\j})
      \cdot
      (-\sin\theta\,\uvec{\i}+\cos\theta\,\uvec{\j})
      = \sin^2\theta + \cos^2\theta = 1
  \end{align}
\end{subequations}
}

\item De los vectores $\vvv{e}_r$ y $\vvv{e}_\theta$
\begin{subequations}
  \begin{align}\label{eq:R2-erer-productoescalar}
    \vvv{e}_r\cdot\vvv{e}_r
    &=
      \xhat{r}\cdot\vvv{r} = 1\\
    \label{eq:R2-eretheta-productoescalar}
    \vvv{e}_r\cdot\vvv{e}_\theta
    &=
      \vvv{e}_\theta\cdot\vvv{e}_r
      = \xhat{r}\cdot r\xhat{\theta} = r \xhat{r}\cdot\xhat{\theta} = 0\\
    \label{eq:R2-ethetaetheta-productoescalar}
    \vvv{e}_r\cdot\vvv{e}_\theta
    &=
      \vvv{e}_\theta\cdot\vvv{e}_\theta
      = r\xhat{\theta}\cdot r\xhat{\theta}
      = r^2 \xhat{\theta}\cdot\xhat{\theta} = r^2\\    
  \end{align}
\end{subequations}
\end{itemize}

\subsubsection{Derivadas de los vectores unitarios}
Para terminar, nos interesan las derivadas de los vectores $\xhat{r}$ y $\xhat{\theta}$
con respecto de $r$
\begin{subequations}
\begin{align}
  \dfrac{d\xhat{r}}{dr}
  &= 0\\
  \dfrac{d\xhat{\theta}}{dr}
  &= 0
\end{align}
\end{subequations}
Derivadas con respecto de $\theta$
\begin{subequations}
\begin{align}
  \dfrac{d\xhat{r}}{d\theta}
  &= -\sin\theta\,\uvec{\i} + \cos\theta\,\uvec{\j} = \xhat{\theta}\\
  \dfrac{d\xhat{\theta}}{d\theta}
  &= -\cos\theta\,\uvec{\i} - \sin\theta\,\uvec{\j} = -\xhat{r}
\end{align}
\end{subequations}

\subsection{Métrica y factores de escala}
Vamos a calcular la métrica y los factores de escala de varias formas:
\begin{enumerate}
\item Directamente, utilizando los productos escalares
  \eqref{eq:R2-erer-productoescalar}, \eqref{eq:R2-eretheta-productoescalar} y
   \eqref{eq:R2-ethetaetheta-productoescalar}
  \begin{equation}\label{eq:R2-metrica-polares}
    \mmm{g}
    = g_{ij}
    = \begin{pmatrix}
      g_{11} & g_{12}\\
      g_{21} & g_{22}
      \end{pmatrix}
    = \begin{pmatrix}
      \vvv{e}_r\cdot\vvv{e}_r & \vvv{e}_r\cdot\vvv{e}_\theta\\
      \vvv{e}_\theta\cdot\vvv{e}_r & \vvv{e}_\theta\cdot\vvv{e}_\theta
    \end{pmatrix}
    = \begin{pmatrix}
      1 & 0\\
      0 & r^2
    \end{pmatrix}
  \end{equation}
  
\item Empezamos calculando las diferenciales $dx$ y $dy$
  {\small
    \begin{subequations}
      \begin{align}\label{eq:R2-dx}
        dx
        &=
          \dfrac{\partial x}{\partial r} dr + \dfrac{\partial x}{\partial\theta} d\theta
          = \cos\theta dr - r\sin\theta d\theta\\
        \label{eq:R2-dy}
        dy
        &=
          \dfrac{\partial y}{\partial r} dr + \dfrac{\partial y}{\partial\theta} d\theta
          = \sin\theta dr + r\cos\theta d\theta
      \end{align}
    \end{subequations}
  }

  La métrica se puede deducir a partir del cuadrado de la distancia recorrida en el
  espacio al cambiar las coordenadas
  {\footnotesize
    \begin{align*}
      ds^2
      &=
        dx^2 + dy^2
        = (\cos\theta dr - r\sin\theta d\theta)^2
        + (\sin\theta dr + r\cos\theta d\theta)^2\\
      &=
        \cos^2\theta dr^2
        + \!r^2\sin^2\theta d\theta^2
        - \!\cancelout{2r\sin\theta \cos\theta dr d\theta}
        + \!\sin^2\theta dr^2
        + \!r^2 \cos^2\theta d\theta^2
        + \!\cancelout{2r\sin\theta \cos\theta dr d\theta}\\
      &=
        (\cos^2\theta + \sin^2\theta) dr^2 + r^2(\sin^2\theta + \cos^2\theta) d\theta^2
        = dr^2 + r^2 d\theta^2
    \end{align*}
  }
  donde se han tenido en cuenta las ecuaciones~\eqref{eq:R2-dx} y ~\eqref{eq:R2-dy} y que
  los vectores unitarios $\xhat{r}$ y $\xhat{\theta}$ son ortogonales.

\item De forma intuitiva podemos interpretar la métrica. Veamos
  \begin{itemize}
  \item Las variables $r$ y $\theta$ son ortogonales, un cambio en una de ellas no afecta
    a la otra, como se puede comprobar observando las figuras
    \ref{fig:R2-polares-intuicion-h-r} y \ref{fig:R2-polares-intuicion-h-theta}.
    De esta manera, vemos que los elementos no diagonales de la métrica valen cero,
    $g_{12}=g_{21}=0$.
  \item Por otro lado, en la figura \ref{fig:R2-polares-intuicion-h-r} observamos que
    cuando modificamos la variable $r$ en una cierta distancia $\Delta r$, estamos
    desplazándonos en el espacio la misma distancia $\Delta s$. esto significa que
    el coeficiente de escala $h_r$ vale la unidad. Esto implica que el elemento
    $g_{11}$ de la métrica vale lo mismo.
    \begin{subequations}
      \begin{align}
        h_r
        &=
          \dfrac{\Delta s}{\Delta r} = 1\\[2pt]
        g_{11}
        &=
          h_x^2 = 1
      \end{align}
    \end{subequations}

  \item En cuanto al coeficiente de escala $h_\theta$, cuando se modifica $\theta$ en una
    cierta cantidad $\Delta\theta$, la distancia que se recorre en el espacio es
    el ángulo por el radio, $\Delta s = r\Delta\theta$. Además, $g_{22} = r^2$. Ver
    figura \ref{fig:R2-polares-intuicion-h-theta}
    \begin{subequations}
      \begin{align}
        h_\theta
        &=
          \dfrac{\Delta s}{\Delta\theta} = \dfrac{r\Delta\theta}{\Delta\theta} = r\\[2pt]
        g_{22}
        &=
          h_\theta^2 = r^2
      \end{align}
    \end{subequations}
    \begin{figure}[ht]
      \centering
      \begin{minipage}{0.40\linewidth}
        % Esta gráfica muestra cómo afecta un cambio en la coordenada r a la distancia
        % recorrida en el espacio R^2
        % Escala
        \def\scl{1.35}
        % 
        \pgfmathsetmacro{\EJESEXTRA}{-0.2}
        \pgfmathsetmacro{\EJESLONG}{2.5}
        % Eje x
        \pgfmathsetmacro{\XMLONG}{\EJESEXTRA}
        \pgfmathsetmacro{\XPLONG}{\EJESLONG}
        % Eje y
        \pgfmathsetmacro{\YMLONG}{\EJESEXTRA}
        \pgfmathsetmacro{\YPLONG}{\EJESLONG}
        % Punto P original
        \pgfmathsetmacro{\PMOD}{1.5}
        \pgfmathsetmacro{\PANG}{30}
        % Punto P final conservando el mismo ángulo
        \pgfmathsetmacro{\PFINMOD}{2.4}
        \pgfmathsetmacro{\PFINANG}{\PANG}      
        % Fondo
        \pgfmathsetmacro{\HORZ}{0.25}
        \pgfmathsetmacro{\VERT}{0.25}
        % 
        \centering
        \begin{tikzpicture}[%
          scale=\scl,
          every node/.style={black,font=\small},
          eje/.style={->},
          r/.style={line width=.8pt, black},
          rtxt/.style={black},
          rfin/.style={line width=1pt, \colorVred},
          rfintxt/.style={\colorTchg, rotate=\PANG},
          textooriginal/.style={\colorTorig},
          pcirculo/.style={black, radius=1pt},
          angulogirado/.style={fill=\colorAng, draw=\colorGirodos},
          background/.style={%
            line width=\bgborderwidth,
            draw=\bgbordercolor,
            fill=\bgcolor,
          },
          ]
          % COORDENADAS
          % Origen
          \coordinate (O) at (0,0);
          % Extremo izdo del eje x e inferior del eje y
          \coordinate (xini) at (\XMLONG cm,0);
          \coordinate (yini) at (0,\YMLONG cm);
          % Eje x
          \path[save path=\ejex] (xini) -- coordinate[pos=0.5] (xtexto)
          (\XPLONG cm, 0) coordinate (xfin);
          % Eje y
          \path[save path=\ejey] (yini) -- coordinate[pos=0.6] (ytexto)
          (0, \YPLONG cm) coordinate (yfin);
          % Punto P original
          \path[save path=\porig] (O) -- coordinate[pos=0.65] (rorigtxt)
          (\PANG:\PMOD cm) coordinate (P);
          % Punto P final
          \path[save path=\pfin] (P) -- coordinate[pos=0.5] (rfintxt)
          (\PFINANG:\PFINMOD cm) coordinate (P');
        
          % DIBUJOS
          % Ángulo theta
          \path (xfin) -- (O) -- (P) pic
          [-{Latex[width=2.5pt,length=3.7pt]},angulogirado,
          "\scriptsize $\theta$",angle radius=24pt, angle eccentricity=0.72]
          {angle = xfin--O--P};
          % Ejes
          % x
          \draw[eje, use path=\ejex];
          \node[right] at (xfin) {$x$};
          % y
          \draw[eje, use path=\ejey];
          \node[above] at (yfin) {$y$};
          % Coordenada r original
          \draw[r, use path=\porig];
          \node[rtxt, above left=-3pt and -1pt] at (rorigtxt) {\footnotesize $r$};
          % Vector r final
          \draw[rfin, use path=\pfin];
          \node[rfintxt, above] at (rfintxt) {\footnotesize $\Delta s = \Delta r$};
          % Puntos
          \fill[pcirculo] (P) circle;
          \fill[pcirculo] (P') circle;
          \node[rtxt, below right] at (P) {\scriptsize $P$};
          \node[rtxt, below right] at (P') {\scriptsize $P'$};
          % Origen
          \filldraw (O) circle [radius=.4pt];
          % Fondo amarillo
          \coordinate (SW)
          at ($(current bounding box.south west) + (-\HORZ cm,-\VERT cm)$);
          \coordinate (NE)
          at ($(current bounding box.north east) + (\HORZ cm,\VERT cm)$);
          \begin{scope}[on background layer]
            \draw[background] (SW) rectangle (NE);
          \end{scope}
        \end{tikzpicture}
        \caption{Al cambiar la coordenada $r$, manteniendo constante $\theta$, se recorre
          una distancia igual en el espacio $\symbb{R}^2$. Por tanto, $h_r=1$ y
          $g_{11} = 1$.}
        \label{fig:R2-polares-intuicion-h-r}
      \end{minipage}
      \hspace{1em}
      \begin{minipage}{0.40\linewidth}
        % Escala
        \def\scl{1.35}
        % 
        \pgfmathsetmacro{\EJESEXTRA}{-0.2}
        \pgfmathsetmacro{\EJESLONG}{2.5}
        % Eje x
        \pgfmathsetmacro{\XMLONG}{\EJESEXTRA}
        \pgfmathsetmacro{\XPLONG}{\EJESLONG}
        % Eje y
        \pgfmathsetmacro{\YMLONG}{\EJESEXTRA}
        \pgfmathsetmacro{\YPLONG}{\EJESLONG}
        % Punto P original
        \pgfmathsetmacro{\PMOD}{2.0}
        \pgfmathsetmacro{\PANG}{25}
        % Punto P final conservando el mismo radio
        \pgfmathsetmacro{\PFINMOD}{\PMOD}
        \pgfmathsetmacro{\PFINANG}{70}
        % Texto en arco
        \pgfmathsetmacro{\PARCANG}{0.5*(\PFINANG) + \PANG)}
        \pgfmathsetmacro{\PARCMOD}{\PMOD * 1.2}
        % Fondo
        \pgfmathsetmacro{\HORZ}{0.25}
        \pgfmathsetmacro{\VERT}{0.25}
        % 
        \centering
        \begin{tikzpicture}[%
          scale=\scl,
          every node/.style={black,font=\small},
          eje/.style={->},
          r/.style={line width=.8pt, black},
          rtxt/.style={black},
          rfin/.style={line width=1pt, \colorVred},
          rfintxt/.style={\colorTchg},
          textooriginal/.style={\colorTorig},
          pcirculo/.style={black, radius=1pt},
          angulo/.style={fill=\colorAng, draw=\colorGirodos},
          angulogirado/.style={fill=\colorAngPink, draw=\colorGiroRed},
          arco/.style={line width=1pt, draw=\colorGiroRed},
          background/.style={%
            line width=\bgborderwidth,
            draw=\bgbordercolor,
            fill=\bgcolor,
          },
          ]

          % COORDENADAS
          % Origen
          \coordinate (O) at (0,0);
          % Extremo izdo del eje x e inferior del eje y
          \coordinate (xini) at (\XMLONG cm,0);
          \coordinate (yini) at (0,\YMLONG cm);
          % Eje x
          \path[save path=\ejex] (xini) -- coordinate[pos=0.5] (xtexto)
          (\XPLONG cm, 0) coordinate (xfin);
          % Eje y
          \path[save path=\ejey] (yini) -- coordinate[pos=0.6] (ytexto)
          (0, \YPLONG cm) coordinate (yfin);
          % Punto P original
          \path[save path=\porig] (O) -- coordinate[pos=0.65] (rorigtxt)
          (\PANG:\PMOD cm) coordinate (P);
          % Punto P final
          \path[save path=\pfin] (O) -- coordinate[pos=0.65] (rfintxt)
          (\PFINANG:\PFINMOD cm) coordinate (P');

          % DIBUJOS
          % Ángulo theta
          \path (xfin) -- (O) -- (P) pic
          [angulo, "\scriptsize $\theta$",angle radius=24pt, angle eccentricity=0.72]
          {angle = xfin--O--P};
          % Incremento de theta
          \path (P) -- (O) -- (P') pic
          [-{Latex[width=3.0pt,length=4.5pt]}, angulogirado,
          "\scriptsize $\Delta\theta$",angle radius=24pt, angle eccentricity=0.72]
          {angle = P--O--P'};
          % Ejes
          % x
          \draw[eje, use path=\ejex];
          \node[right] at (xfin) {$x$};
          % y
          \draw[eje, use path=\ejey];
          \node[above] at (yfin) {$y$};
          % Longitud r original
          \draw[r, use path=\porig];
          \node[rtxt, below right=-2pt and 0pt] at (rorigtxt) {\footnotesize $r$};
          % Longitud r final
          \draw[r, use path=\pfin];
          \node[rtxt, above left = -2pt and 0pt] at (rfintxt) {\footnotesize $r$};
          % Arco recorrido
          \draw[arco] (P) arc (\PANG:\PFINANG:\PMOD)
          node[rfintxt,above,midway,sloped] {\footnotesize$\Delta s = r\Delta\theta$};
        
          % Puntos
          \fill[pcirculo] (P) circle;
          \fill[pcirculo] (P') circle;
          \node[rtxt, below right=-3pt and 0pt] at (P) {\scriptsize $P$};
          \node[rtxt, above left=0pt and -3pt] at (P') {\scriptsize $P'$};
          % Origen
          \filldraw (O) circle [radius=.4pt];
          % Fondo amarillo
          \coordinate (SW)
          at ($(current bounding box.south west) + (-\HORZ cm,-\VERT cm)$);
          \coordinate (NE)
          at ($(current bounding box.north east) + (\HORZ cm,\VERT cm)$);
          \begin{scope}[on background layer]
            \draw[background] (SW) rectangle (NE);
          \end{scope}

        \end{tikzpicture}
        \caption{Al cambiar la coordenada $\theta$, manteniendo $r$ constante, se  recorre
          una distancia proporcional a $r$ y a $\Delta\theta$. Por tanto, $h_\theta = r$ y
          $g_{22} = r^2$.}
        \label{fig:R2-polares-intuicion-h-theta}
      \end{minipage}
    \end{figure}
  \end{itemize}
  
\end{enumerate}

\subsection{Gradiente}
Cálculos preliminares. Se utilizan las expresiones \eqref{eq:R2-x-rtheta},
\eqref{eq:R2-y-rtheta}, \eqref{eq:R2-r-xy} y \eqref{eq:R2-theta-xy}
{\small
\begin{subequations}
  \begin{align}
    \dfrac{\partial r}{\partial x}
    &=
      \dfrac{1}{\cancelout{2}\sqrt{x^2+y^2}}\cdot \cancelout{2}x
      = \dfrac{\cancelout{r}\cos\theta}{\cancelout{r}} = \cos\theta\\
    \dfrac{\partial r}{\partial y}
    &=
      \dfrac{1}{\cancelout{2}\sqrt{x^2+y^2}}\cdot \cancelout{2}y
      = \dfrac{\cancelout{r}\sin\theta}{\cancelout{r}} = \sin\theta\\
    \dfrac{\partial\theta}{\partial x}
    &=
      \dfrac{1}{1+\dfrac{y^2}{x^2}}\,\left(-\dfrac{y}{x}\right)
      = -\dfrac{x^{\cancelout{\scriptstyle{2}}}}{x^2+y^2}\,\dfrac{y}{\cancelout{x}}
      = -\dfrac{y}{x^2+y^2}
      = -\dfrac{\cancelout{r}\sin\theta}{r^{\cancelout{2}}}
      = -\dfrac{\sin\theta}{r}\\
    \dfrac{\partial\theta}{\partial x}
    &=
      \dfrac{1}{1+\dfrac{y^2}{x^2}}\,\dfrac{1}{x}
      = \dfrac{x^{\cancelout{\scriptstyle{2}}}}{x^2+y^2}\,\dfrac{1}{\cancelout{x}}
      = \dfrac{y}{x^2+y^2}
      = \dfrac{\cancelout{r}\cos\theta}{r^{\cancelout{\scriptstyle{2}}}}
      = \dfrac{\cos\theta}{r}
  \end{align}
\end{subequations}
}
{\small
  \begin{subequations}
    \begin{align*}
      \dfrac{\partial f}{\partial x}\,\uvec{\i}
      &=
        \left(%
        \dfrac{\partial f}{\partial r}\dfrac{\partial r}{\partial x}
        + \dfrac{\partial f}{\partial\theta}\dfrac{\partial\theta}{\partial x}
        \right)
        \left(%
        \cos\theta\,\xhat{r}-\sin\theta\,\xhat{\theta}
        \right)\\
      &=
        \left[%
        \dfrac{\partial f}{\partial r}\cos\theta
        + \dfrac{\partial f}{\partial\theta}\left(-\dfrac{\sin\theta}{r}\right)
        \right]
        \left(%
        \cos\theta\,\xhat{r} - \sin\theta\,\xhat{\theta}
        \right)\\
      &=
        \cos^2\theta\dfrac{\partial f}{\partial r}\,\xhat{r}
        -\dfrac{\sin\theta\cos\theta}{r}\dfrac{\partial f}{\partial\theta}\,\xhat{r}
        - \sin\theta\cos\theta\dfrac{\partial f}{\partial r}\,\xhat{\theta}
        + \dfrac{\cos^2\theta}{r}\dfrac{\partial f}{\partial\theta}\,\xhat{\theta}
    \end{align*}
  \end{subequations}
}
{\small
  \begin{subequations}
    \begin{align*}
      \dfrac{\partial f}{\partial y}\,\uvec{\j}
      &=
        \left(%
        \dfrac{\partial f}{\partial r}\dfrac{\partial f}{\partial y}
        + \dfrac{\partial f}{\partial\theta}\dfrac{\partial\theta}{\partial y}
        \right)
        \left(%
        \sin\theta\,\xhat{r}-\cos\theta\,\xhat{\theta}
        \right)\\
      &=
        \left[%
        \dfrac{\partial f}{\partial r}\sin\theta
        + \dfrac{\partial f}{\partial\theta}\dfrac{\cos\theta}{r}
        \right]
        \left(%
        \sin\theta\,\xhat{r} - \cos\theta\,\xhat{\theta}
        \right)\\
      &=
        \sin^2\theta\dfrac{\partial f}{\partial r}\,\xhat{r}
        + \dfrac{\sin\theta\cos\theta}{r}\dfrac{\partial f}{\partial\theta}\,\xhat{r}
        + \sin\theta\cos\theta\dfrac{\partial f}{\partial r}\,\xhat{\theta}
        + \dfrac{\cos^2\theta}{r}\dfrac{\partial f}{\partial\theta}\,\xhat{\theta}
    \end{align*}
  \end{subequations}
}

El gradiente en polares se obtiene sustituyendo las dos expresiones anteriores en
\eqref{eq:R2-gradiente-cartesianas} y sumando. A continuación presentamos el resultado
{\small
  \begin{equation}
    \vvv{\nabla}\cdot f
    =
      \dfrac{\partial f}{\partial x}\,\uvec{\i}
      + \dfrac{\partial f}{\partial y}\,\uvec{\j}
      = \dfrac{\partial f}{\partial r}\,\xhat{r}
      + \dfrac{1}{r}\dfrac{\partial f}{\partial\theta}\,\xhat{\theta}
  \end{equation}
}

\subsection{Divergencia}
La divergencia de un campo vectorial en polares la calcularemos de varias formas pero,
sin que sirva de precedente, la primera que presentaré no es correcta:
\begin{enumerate}
\item En este producto escalar se utiliza la métrica en polares
  \eqref{eq:R2-metrica-polares}
  \begin{equation}
    \vvv{\nabla}\cdot\vvv{A}
    \neq
    \cancelout{%
        \begin{pmatrix}
          \partial/\partial r & \partial/\partial\theta      
        \end{pmatrix}
        \begin{pmatrix}
          1 & 0\\
          0 & r^2
        \end{pmatrix}
        \begin{pmatrix}
          A_r\\
          A_\theta
        \end{pmatrix}
      }
      = \cancelout{%
        \dfrac{\partial A_r}{\partial r}
        + \dfrac{\partial r^2A_\theta}{\partial\theta}
      }
    \end{equation}
  ¿Por qué no es correcto este cálculo, si se ha utilizado la métrica correcta en
  polares, de forma similar a lo que se hizo en cartesianas, ver
  \eqref{eq:R2-divergencia-cartesianas}?
  
  %\hspace{1em}
  El problema se debe a que, aunque hemos utilizado la métrica, esta solo corrige el
  factor de escala en coordenadas polares. Pero en este cálculo matricial, y en el que se
  realizó en cartesianas, se supone implícitamente que los vectores de la base son
  inmutables, lo que es correcto en cartesianas pero no en polares, dado que los
  vectores cambian con la posición en el plano $\symbb{R}^2$.
  En resumen, el sistema de coordenadas polares es curvilíneo y esa curvatura hay que
  tenerla en cuenta, ver figura \ref{fig:R2-coordenadas-polares}.
  \begin{figure}[ht]
  % Escala
  \def\scl{0.9}
  % 
  % Número de círcunferencias
  \pgfmathsetmacro{\NUMCIRC}{5}
  % Longitud de un versor
  \pgfmathsetmacro{\VERSOR}{0.5}
  % Longitud de línea radial
  \pgfmathsetmacro{\LINEARADIAL}{\NUMCIRC * \VERSOR + 0.25}
  % Fondo
  \pgfmathsetmacro{\HORZ}{0.25}
  \pgfmathsetmacro{\VERT}{0.25}
  % 
  \centering
  \begin{tikzpicture}[%
    scale=\scl,
    every node/.style={black,font=\small},
    vtexto/.style={font=\footnotesize, pos=1.4},
    vtheta/.style={-{Latex[round, width=4pt, length=5pt]},
      line width=0.6pt, \colorVred},
    vr/.style={vtheta, \colorV},
    polares/.style={draw=black!20, line width=0.4pt},
    punto/.style={fill=black, radius=1.25pt},
    background/.style={%
      line width=\bgborderwidth,
      draw=\bgbordercolor,
      fill=\bgcolor
    },
    ]
    % COORDENADAS
    % Origen
    \coordinate (O) at (0,0);

      % DIBUJOS
      % Coordenadas polares (Radios de los ángulos)
      \foreach \angulo / \label
      in {%
        0/{0}, 30/{\pi/6}, 60/{\pi/3},
        90/{\pi/2}, 120/{2\pi/3}, 150/{5\pi/6},
        180/{\pi}, 210/{7\pi/6}, 240/{4\pi/3},
        270/{3\pi/2}, 300/{5\pi/3}, 330/{11\pi/6}
      }
      {%
        \draw[polares] (O) -- ++(\angulo:\LINEARADIAL)
        node[pos=1.12] {\scriptsize $\label$};
      };
      \foreach \radio [evaluate=\radio] in {\VERSOR*1,\VERSOR*...,\VERSOR*\NUMCIRC}
      {%
        % Circunferencias
        \draw[polares] (O) circle[radius=\radio cm];
      };

      % Vectores unitarios en r=1.0
      \foreach \radio [evaluate=\radio] in{\VERSOR*1}
      \foreach \angulo in {30}
      {%
        % Vectores unitario radiales
        \draw[vr] (O) ++(\angulo:\radio) -- ++(\angulo:\VERSOR)
        node[vtexto, rotate=\angulo-90] {$\xhat{r}$};
        % Vectores unitarios angulares
        \draw[vtheta] (O) ++(\angulo:\radio) -- ++(90+\angulo:\VERSOR)
        node[vtexto, rotate=\angulo-90, \colorTred] {$\xhat{\theta}$};
        % Posición
        \fill[punto] (O) ++(\angulo:\radio) circle;
      };
      
      
      % Vectores unitarios en r=2.0
      \foreach \radio [evaluate=\radio] in{\VERSOR*2}
      \foreach \angulo in {150}
      {%
        % Vectores unitario radiales
        \draw[vr] (O) ++(\angulo:\radio) -- ++(\angulo:\VERSOR)
        node[vtexto, rotate=\angulo-90] {$\xhat{r}$};
        % Vectores unitarios angulares
        \draw[vtheta] (O) ++(\angulo:\radio) -- ++(90+\angulo:\VERSOR)
        node[vtexto, rotate=\angulo-90, \colorTred] {$\xhat{\theta}$};
        % Posición
        \fill[punto] (O) ++(\angulo:\radio) circle;
      };
      
      % Vectores unitarios en r=3.0
      \foreach \radio [evaluate=\radio] in{\VERSOR*3}
      \foreach \angulo in {210, 330}
      {%
        % Vectores unitario radiales
        \draw[vr] (O) ++(\angulo:\radio) -- ++(\angulo:\VERSOR)
        node[vtexto, rotate=\angulo-90] {$\xhat{r}$};
        % Vectores unitarios angulares
        \draw[vtheta] (O) ++(\angulo:\radio) -- ++(90+\angulo:\VERSOR)
        node[vtexto, rotate=\angulo-90, \colorTred] {$\xhat{\theta}$};
        % Posición
        \fill[punto] (O) ++(\angulo:\radio) circle;
      };

      % Vectores unitarios en r=4.0
      \foreach \radio [evaluate=\radio] in{\VERSOR*4}
      \foreach \angulo in {60,270}
      {%
        % Vectores unitario radiales
        \draw[vr] (O) ++(\angulo:\radio) -- ++(\angulo:\VERSOR)
        node[vtexto, rotate=\angulo-90] {$\xhat{r}$};
        % Vectores unitarios angulares
        \draw[vtheta] (O) ++(\angulo:\radio) -- ++(90+\angulo:\VERSOR)
        node[vtexto, rotate=\angulo-90, \colorTred] {$\xhat{\theta}$};
        % Posición
        \fill[punto] (O) ++(\angulo:\radio) circle;
      };      
      
      % Origen
      \filldraw (O) circle [radius=.2pt];
    
      % Fondo amarillo
      \coordinate (SW) at ($(current bounding box.south west) + (-\HORZ cm,-\VERT cm)$);
      \coordinate (NE) at ($(current bounding box.north east) + (\HORZ cm,\VERT cm)$);
      \begin{scope}[on background layer]
        \draw[background] (SW) rectangle (NE);
      \end{scope}
    \end{tikzpicture}
    \caption{Coordenadas polares (curvilíneas), donde $r$ representa la distancia al
      origen (polo) y $\theta$ es el ángulo que forma el vector de posición con el eje
      $x$, aquí expresado en radianes. Además se muestran algunos vectores unidad
      $\xhat{r}$ y $\xhat{\theta}$ en distintas posiciones del plano $\symbb{R}^2$.}
    \label{fig:R2-coordenadas-polares}
  \end{figure}

\item Segundo método
  
\end{enumerate}


\subsection{Laplaciana} 

\subsection{Elemento de área en coordenadas polares planas}
\begin{align*}
  d^2a
  &=
    \dfrac{\partial(x,y)}{\partial(r,\theta)} dr\,d\theta
  = \begin{vmatrix}
    \partial x/\partial r & \partial x/\partial\theta\\
    \partial y/\partial r & \partial y/\partial\theta
  \end{vmatrix}
  dr\,d\theta
  = \begin{vmatrix}
    \cos\theta & -r\sin\theta\\
    \sin\theta & r\cos\theta
  \end{vmatrix}
    dr\,d\theta\\
  &=
    (r\cos^2\theta + r\sin^2\theta)\,dr\,d\theta
    = r (\cos^2\theta + \sin^2\theta)\,dr\,d\theta
\end{align*}
donde hemos utilizado las relaciones \eqref{eq:R2-x-rtheta} y \eqref{eq:R2-y-rtheta}.
Simplificando finalmente, queda
\begin{equation}
  d^2a = rdr\,d\theta
\end{equation}



%%% Local Variables:
%%% mode: latex
%%% TeX-engine: luatex
%%% TeX-master: "../retazosmatematicas.tex"
%%% End:
