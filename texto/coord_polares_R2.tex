% metrica_euclid_ortog.tex
%
% Copyright (C) 2022--2026 José A. Navarro Ramón <janr.devel@gmail.com>
% Licencia del código GPLv2
% Licencia Creative Commons Recognition Non-Commercial Share-alike.
% (CC-BY-NC-SA)

\chapter{Coordenadas polares planas (espacio euclídeo
  \mathinhead{\symbb{R}^2}{espr2})}
\section{Coordenadas cartesianas}

\subsection{Gradiente, divergencia y laplaciana}
Presentamos un resumen de estas operaciones
\begin{itemize}
\item El gradiente de un campo escalar $\phi(x,y)$ es un vector
\begin{equation}
  \vvv{\nabla}\phi
  = \dfrac{\partial \phi}{\partial x}\,\uvec{\i}
  + \dfrac{\partial \phi}{\partial y}\,\uvec{\j}
  = \begin{pmatrix}
    \partial \phi/\partial x\\
    \partial \phi/\partial y
    \end{pmatrix}
\end{equation}
El vector gradiente de $\phi$ evaluado en un punto genérico $(x,y)$ del dominio de $\phi$
indica la dirección en la cual el campo $\phi$ varía más rápidamente y su módulo
representa el ritmo de variación de $\phi$ en la dirección de dicho vector gradiente.

El operador gradiente se comporta como un vector bajo rotaciones
\begin{equation}
  \vvv{\nabla}
  = \dfrac{\partial}{\partial x}\,\uvec{\i}
  + \dfrac{\partial}{\partial y}\,\uvec{\j}
  = \begin{pmatrix}
    \partial/\partial x\\
    \partial/\partial y
    \end{pmatrix}
\end{equation}

\item La divergencia de un campo vectorial $\vvv{A}=(A_x, A_y)$ es un escalar
  \begin{equation}
    \vvv{\nabla}\cdot\vvv{A}
    = \dfrac{\partial A_x}{\partial x}
    + \dfrac{\partial A_y}{\partial y}
  \end{equation}
  La divergencia mide la diferencia entre el flujo saliente y el flujo entrante de un
  campo vectorial sobre la superficie que rodea a un volumen de control, por tanto, si el
  campo tiene ''fuentes'' la divergencia será positiva, y si tiene "sumideros", la
  divergencia será negativa. La divergencia mide la rapidez neta con la que se conduce la
  materia al exterior de cada punto, y en el caso de ser la divergencia idénticamente
  igual a cero, describe al flujo incompresible del fluido, llamado también campo
  solenoidal.
  
\item La laplaciana de un campo escalar $f(x,y)$ es la divergencia del gradiente y es un
  escalar
  \begin{equation}
    \Delta \phi
    =
    \nabla^2 \phi
    = \vvv{\nabla}\cdot\vvv{\nabla}\phi
    = \dfrac{\partial^2 \phi}{\partial x^2}
    + \dfrac{\partial^2 \phi}{\partial y^2}
  \end{equation}
Mide la concentración de un campo en un punto. Si es negativo, indica una tendencia a la
concentración (fuente); si es positivo, indica dispersión.

El operador laplaciano es
\begin{equation}
  \nabla^2
  = \dfrac{\partial^2}{\partial x^2}
  + \dfrac{\partial^2}{\partial y^2}
\end{equation}
\end{itemize}

\subsection{Métrica y factores de escala}
Empezamos calculando el producto escalar de dos vectores, expresándolo en forma matricial
\begin{align*}
  \vvv{r}_1 &= a\,\uvec{\i} + b\,\uvec{\j}\\
  \vvv{r}_2 &= c\,\uvec{\i} + d\,\uvec{\j}
\end{align*}
\begin{align*}
  \vvv{r}_1\cdot\vvv{r}_2
  &=
    (a\,\uvec{\i} + b\,\uvec{\j}) \cdot (c\,\uvec{\i} + d\,\uvec{\j})
    = ac\,\uvec{\i}\cdot\uvec{i} + ad\,\uvec{\i}\cdot\uvec{\j}
    + bc\,\uvec{\j}\cdot\uvec{\i} + bd\,\uvec{\j}\cdot\uvec{\j}\\
  &=
    \begin{pmatrix}
      a & b
    \end{pmatrix}
    \begin{pmatrix}
      \uvec{\i}\cdot\uvec{\i} & \uvec{\i}\cdot\uvec{\j}\\
      \uvec{i}\cdot\uvec{j} & \uvec{\j}\cdot\uvec{\j}
    \end{pmatrix}
    \begin{pmatrix}
      c\\
      d
    \end{pmatrix}
    =
    \begin{pmatrix}
      a & b
    \end{pmatrix}
    \begin{pmatrix}
      1 & 0\\
      0 & 1
    \end{pmatrix}
    \begin{pmatrix}
      c\\
      d
    \end{pmatrix}\\
  &=
    \begin{pmatrix}
      a & b
    \end{pmatrix}
    \begin{pmatrix}
      c\\
      d
    \end{pmatrix}
    =
    ab + cd
\end{align*}

La matriz $2\times 2$ que aparece en el cálculo es la \emph{métrica} del espacio
$\symbb{R}^2$ en coordenadas cartesianas
\begin{equation}\label{eq:R2-metrica-cartesianas}
  \mmm{g}
  =
  g_{ij}
  =
  \begin{pmatrix}
      \uvec{\i}\cdot\uvec{\i} & \uvec{\i}\cdot\uvec{\j}\\
      \uvec{i}\cdot\uvec{j} & \uvec{\j}\cdot\uvec{\j}    
  \end{pmatrix}
  = 
  \begin{pmatrix}
    g_{11} & g_{12}\\
    g_{21} & g_{22}
  \end{pmatrix}
  =
  \begin{pmatrix}
    1 & 0\\
    0 & 1
  \end{pmatrix}
\end{equation}

Se llaman \emph{factores de escala} o \emph{coeficientes métricos} a la raíz cuadrada de
los elementos diagonales en la métrica en un sistema de coordenadas ortogonales, como el
que nos ocupa. En coordenadas cartesianas estándar serían
\begin{align*}
  h_x &= \sqrt{g_{11}} = \sqrt{1} = 1\\
  h_y &= \sqrt{g_{22}} = \sqrt{1} = 1  
\end{align*}

Fijémonos en la métrica \eqref{eq:R2-metrica-cartesianas}
\begin{itemize}
\item Los elementos no diagonales son nulos. Esto se interpreta como que los vectores
  unitarios $\uvec{\i}$ y $\uvec{\j}$ son independientes, esto es, son ortogonales.
  Esto es, un cambio en una coordenada no afecta a la otra.
\item Los elementos diagonales valen la unidad. Cada elemento diagonal de la métrica que
  valga la unidad implica que un cambio en una unidad en la coordenada respectiva implica
  una distancia recorrida de una unidad en el espacio. En realidad, el espacio recorrido
  es el factor de escala correspondiente, pero en este caso da lo mismo
  \begin{subequations}
    \begin{align}
      \Delta s &= h_x \Delta x = \sqrt{1}\, \Delta x = \Delta x\\
      \Delta s &= h_y \Delta y = \sqrt{1}\, \Delta y = \Delta y
    \end{align}
  \end{subequations}
\end{itemize}

\section{Coordenadas polares planas}
En este sistema, se utilizan dos coordenadas, $r$ y $\theta$. La primera coordenada es la
distancia entre el origen de coordenadas y un punto del espacio.

\begin{figure}[ht]
  % Escala
  \def\scl{1}
  %
  \pgfmathsetmacro{\EJESEXTRA}{-0.2}
  \pgfmathsetmacro{\EJESLONG}{2.3}
  % Eje x
  \pgfmathsetmacro{\XMLONG}{\EJESEXTRA}
  \pgfmathsetmacro{\XPLONG}{\EJESLONG}
  % Eje y
  \pgfmathsetmacro{\YMLONG}{\EJESEXTRA}
  \pgfmathsetmacro{\YPLONG}{\EJESLONG}
  % Vector P
  \pgfmathsetmacro{\PMOD}{2.5}
  \pgfmathsetmacro{\PANG}{40}
  % Fondo
  \pgfmathsetmacro{\HORZ}{0.25}
  \pgfmathsetmacro{\VERT}{0.25}
  % 
  \centering
  \begin{tikzpicture}[%
    scale=\scl,
    every node/.style={black,font=\small},
    eje/.style={->},
    proyeccion/.style={black!60, densely dotted},
    vector/.style={-{Latex[round]}, shorten >=1.2pt, line width=.8pt,\colorV},
    textooriginal/.style={\colorTorig},
    pcirculo/.style={fill=\colorPorig, draw=black},
    angulogirado/.style={fill=\colorAngGreen, draw=\colorGirodos},
    background/.style={
      line width=\bgborderwidth,
      draw=\bgbordercolor,
      fill=\bgcolor,
    },
    ]
    % COORDENADAS
    % Origen
    \coordinate (O) at (0,0);
    % Extremo izdo del eje x e inferior del eje y
    \coordinate (xini) at (\XMLONG cm,0);
    \coordinate (yini) at (0,\YMLONG cm);
    % Eje x
    \path[save path=\ejex] (xini) -- coordinate[pos=0.5] (xtexto) (\XPLONG cm, 0)
    coordinate (xfin);
    % Eje y
    \path[save path=\ejey] (yini) -- coordinate[pos=0.6] (ytexto) (0, \YPLONG cm)
    coordinate (yfin);
    % Vector
    \path[save path=\vector] (O) -- coordinate[pos=0.5] (pmid) (\PANG:\PMOD cm)
    coordinate (P);
    % Proyección de P en los ejes
    \path[save path=\proyx] (P) |- (xfin);
    \path[save path=\proyy] (P) -| (yfin);

    % DIBUJOS
    % Ángulo theta
    \path (xfin) -- (O) -- (P) pic
    [-{Latex[width=2.5pt,length=3.7pt]},angulogirado,
    "\footnotesize $\theta$",angle radius=20pt, angle eccentricity=0.68]
    {angle = xfin--O--P};
    % Proyección de P en los ejes
    \draw[proyeccion, use path=\proyx];
    \draw[proyeccion, use path=\proyy];
    % Ejes
    % x
    \draw[eje, use path=\ejex];
    \node[right] at (xfin) {$x$};
    \node[below] at (xtexto) {$r\cos\theta$};
    % y
    \draw[eje, use path=\ejey];
    \node[above] at (yfin) {$y$};
    \node[rotate=90, left=8pt] at (ytexto) {$r\sin\theta$};
    % Vector y punto  P
    \draw[vector, use path=\vector];
    \node[above left=-4pt and -2pt] at (pmid) {$\vvv{r}$};
    \filldraw[fill=green, draw=black] (P) circle[radius=1.2pt];
    \node[above] at (P) {$P$};
    
    % Origen
    \filldraw (O) circle [radius=.2pt];
    % Fondo amarillo
    \coordinate (SW) at ($(current bounding box.south west) + (-\HORZ cm,-\VERT cm)$);
    \coordinate (NE) at ($(current bounding box.north east) + (\HORZ cm,\VERT cm)$);    
    \begin{scope}[on background layer]
      \draw[background] (SW) rectangle (NE);
    \end{scope}
  \end{tikzpicture}
  \caption{Coordenadas cartesianas y polares planas de un punto $P$ del plano.}
  \label{fig:R2:coordenadas-R2}
\end{figure}

\subsection{Relación con las coordenadas cartesianas}
De la figura se desprenden las siguientes relaciones
\begin{subequations}
  \begin{align}
    x &= r\cos\theta\\
    y &= r\sin\theta
  \end{align}
\end{subequations}
y, sus inversas
\begin{subequations}
  \begin{align}
    r &= \sqrt{x^2 + y^2}\\
    \theta &= \arctan\frac{y}{x}
  \end{align}
\end{subequations}

\subsection{Vectores unitarios}


\subsection{Métrica y factores de escala}
Empezamos calculando las diferenciales $dx$ y $dy$
\begin{subequations}
  \begin{align}\label{eq:R2-dx}
    dx
    &=
      \dfrac{\partial x}{\partial r} + \dfrac{\partial x}{\partial\theta}
      = \cos\theta dr - r\sin\theta d\theta\\
    \label{eq:R2-dy}
    dy
    &=
      \dfrac{\partial y}{\partial r} + \dfrac{\partial y}{\partial\theta}
      = \sin\theta dr + r\cos\theta d\theta
\end{align}
\end{subequations}

La métrica se puede deducir a partir de la distancia recorrida en el espacio al cambiar
las coordenadas
{\footnotesize
\begin{align*}
  ds^2
  &=
    dx^2 + dy^2
    = (\cos\theta dr - r\sin\theta d\theta)^2 + (\sin\theta dr + r\cos\theta d\theta)^2\\
  &=
    \cos^2\theta dr^2
    + r^2 \sin^2\theta d\theta^2
    - \cancelout{2r\sin\theta \cos\theta dr d\theta}
    + \sin^2\theta dr^2
    + r^2 \cos^2\theta d\theta^2
    + \cancelout{2r\sin\theta \cos\theta dr d\theta}\\
  &=
    (\cos^2\theta + \sin^2\theta) dr^2 + r^2(\sin^2\theta + \cos^2\theta) d\theta^2
    = dr^2 + r^2 d\theta^2
\end{align*}
}
donde se han tenido en cuenta las ecuaciones~\eqref{eq:R2-dx} y ~\eqref{eq:R2-dy}.
Los vectores unitarios $\xhat{r}$ y $\xhat{\theta}$ son ortogonales.

\subsection{Gradiente}

\subsection{Divergencia}

\subsection{Laplaciana} 



%%% Local Variables:
%%% mode: latex
%%% TeX-engine: luatex
%%% TeX-master: "../retazosmatematicas.tex"
%%% End:
