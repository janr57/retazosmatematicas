% metrica_euclid_ortog.tex
%
% Copyright (C) 2022--2026 José A. Navarro Ramón <janr.devel@gmail.com>
% Licencia del código GPLv2
% Licencia Creative Commons Recognition Non-Commercial Share-alike.
% (CC-BY-NC-SA)

\chapter{Coordenadas polares planas (en espacio euclídeo
  \mathinhead{\symbb{R}^2}{espr2})}
\section{Intuición acerca de la métrica en coordenadas cartesianas}
Empezamos con la métrica en coordenadas cartesianas es
\begin{equation}\label{eq:R2-metrica-cartesianas}
  \mmm{g}
  =
  g_{ij}
  =
  \begin{pmatrix}
    g_{11} & g_{12}\\
    g_{21} & g_{22}
  \end{pmatrix}
  =
  \begin{pmatrix}
    1 & 0\\
    0 & 1
  \end{pmatrix}
\end{equation}

Se llaman \emph{factores de escala} o \emph{coeficientes métricos} a la raíz cuadrada de
los elementos diagonales en la métrica en un sistema de coordenadas ortogonales, como el
que nos ocupa. En coordenadas cartesianas estándar serían
\begin{align*}
  h_x &= \sqrt{g_{11}} = \sqrt{1} = 1\\
  h_y &= \sqrt{g_{22}} = \sqrt{1} = 1  
\end{align*}

Fijémonos en la métrica \eqref{eq:R2-metrica-cartesianas}
\begin{itemize}
\item Los elementos no diagonales son nulos. Esto se interpreta como que las coordenadas
  $x$ e $y$ son independientes, esto es, son ortogonales. Esto es, un cambio en una de ellas no afecta a la otra.
\item Los elementos diagonales valen la unidad. Cada elemento diagonal de la métrica que
  valga la unidad implica que un cambio en una unidad en la coordenada respectiva implica
  una distancia recorrida de una unidad en el espacio. En realidad, el espacio recorrido
  es el factor de escala correspondiente, pero en este caso da lo mismo
  \begin{subequations}
    \begin{align}
      \Delta s &= h_x |\Delta x| = \sqrt{1}\, |\Delta x| = |\Delta x|\\
      \Delta s &= h_y |\Delta y| = \sqrt{1}\, |\Delta y| = |\Delta y|
    \end{align}
  \end{subequations}
\end{itemize}


\section{Coordenadas polares planas}
En este sistema, se utilizan dos coordenadas, $r$ y $\theta$. La primera coordenada es la
distancia entre el origen de coordenadas y un punto del espacio.

\begin{figure}[ht]
  % Escala
  \def\scl{1}
  %
  \pgfmathsetmacro{\EJESEXTRA}{-0.2}
  \pgfmathsetmacro{\EJESLONG}{2.3}
  % Eje x
  \pgfmathsetmacro{\XMLONG}{\EJESEXTRA}
  \pgfmathsetmacro{\XPLONG}{\EJESLONG}
  % Eje y
  \pgfmathsetmacro{\YMLONG}{\EJESEXTRA}
  \pgfmathsetmacro{\YPLONG}{\EJESLONG}
  % Vector P
  \pgfmathsetmacro{\PMOD}{2.5}
  \pgfmathsetmacro{\PANG}{40}
  % Fondo
  \pgfmathsetmacro{\HORZ}{0.25}
  \pgfmathsetmacro{\VERT}{0.25}
  % 
  \centering
  \begin{tikzpicture}[%
    scale=\scl,
    every node/.style={black,font=\small},
    eje/.style={->},
    proyeccion/.style={black!60, densely dotted},
    vector/.style={-{Latex[round]}, shorten >=1.2pt, line width=.8pt,\colorV},
    textooriginal/.style={\colorTorig},
    pcirculo/.style={fill=\colorPorig, draw=black},
    angulogirado/.style={fill=\colorAngGreen, draw=\colorGirodos},
    background/.style={
      line width=\bgborderwidth,
      draw=\bgbordercolor,
      fill=\bgcolor,
    },
    ]
    % COORDENADAS
    % Origen
    \coordinate (O) at (0,0);
    % Extremo izdo del eje x e inferior del eje y
    \coordinate (xini) at (\XMLONG cm,0);
    \coordinate (yini) at (0,\YMLONG cm);
    % Eje x
    \path[save path=\ejex] (xini) -- coordinate[pos=0.5] (xtexto) (\XPLONG cm, 0)
    coordinate (xfin);
    % Eje y
    \path[save path=\ejey] (yini) -- coordinate[pos=0.6] (ytexto) (0, \YPLONG cm)
    coordinate (yfin);
    % Vector
    \path[save path=\vector] (O) -- coordinate[pos=0.5] (pmid) (\PANG:\PMOD cm)
    coordinate (P);
    % Proyección de P en los ejes
    \path[save path=\proyx] (P) |- (xfin);
    \path[save path=\proyy] (P) -| (yfin);

    % DIBUJOS
    % Ángulo theta
    \path (xfin) -- (O) -- (P) pic
    [-{Latex[width=2.5pt,length=3.7pt]},angulogirado,
    "\footnotesize $\theta$",angle radius=20pt, angle eccentricity=0.68]
    {angle = xfin--O--P};
    % Proyección de P en los ejes
    \draw[proyeccion, use path=\proyx];
    \draw[proyeccion, use path=\proyy];
    % Ejes
    % x
    \draw[eje, use path=\ejex];
    \node[right] at (xfin) {$x$};
    \node[below] at (xtexto) {$r\cos\theta$};
    % y
    \draw[eje, use path=\ejey];
    \node[above] at (yfin) {$y$};
    \node[rotate=90, left=8pt] at (ytexto) {$r\sin\theta$};
    % Vector y punto  P
    \draw[vector, use path=\vector];
    \node[above left=-4pt and -2pt] at (pmid) {$\vvv{r}$};
    \filldraw[fill=green, draw=black] (P) circle[radius=1.2pt];
    \node[above] at (P) {$P$};
    
    % Origen
    \filldraw (O) circle [radius=.2pt];
    % Fondo amarillo
    \coordinate (SW) at ($(current bounding box.south west) + (-\HORZ cm,-\VERT cm)$);
    \coordinate (NE) at ($(current bounding box.north east) + (\HORZ cm,\VERT cm)$);    
    \begin{scope}[on background layer]
      \draw[background] (SW) rectangle (NE);
    \end{scope}
  \end{tikzpicture}
  \caption{Coordenadas cartesianas y polares planas de un punto $P$ del plano.}
  \label{fig:R2:coordenadas-R2}
\end{figure}

De la figura se desprenden las siguientes relaciones
\begin{subequations}
  \begin{align}
    x &= r\cos\theta\\
    y &= r\sin\theta
  \end{align}
\end{subequations}
y, sus inversas
\begin{subequations}
  \begin{align}
    r &= \sqrt{x^2 + y^2}\\
    \theta &= \arctan\frac{y}{x}
  \end{align}
\end{subequations}

%%% Local Variables:
%%% mode: latex
%%% TeX-engine: luatex
%%% TeX-master: "../retazosmatematicas.tex"
%%% End:
